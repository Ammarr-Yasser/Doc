\section{Overview}
The Link Analyzer evaluates a submitted URL and returns a risk-oriented result. The tool combines URL checks, reputation lookups, and optional deeper inspection paths to estimate whether a link is likely safe or suspicious.

This tool matters for social engineering defense because malicious campaigns often start with a URL. Early link triage helps users and analysts block phishing pages, malware delivery links, and impersonation attempts before account or device compromise.

\begin{table}[H]
  \centering
  \caption{Link Analyzer API Endpoints}
  \begin{tabular}{|l|p{0.62\textwidth}|}
    \hline
    \textbf{Endpoint} & \textbf{Description} \\ \hline
    \texttt{POST /quick-scan} & Fast URL inspection \\ \hline
    \texttt{POST /deep-scan} & Deeper URL inspection with enrichment \\ \hline
    \texttt{POST /custom-scan} & Custom scan options \\ \hline
  \end{tabular}
\end{table}

\begin{figure}[H]
  \centering
  \includegraphics[width=\linewidth]{figures/fig-authentication-jwt-sequence.png}
  \caption{Authentication workflow and JWT-protected request sequence for Link Analyzer execution.}
  \label{fig:link-analyzer-request-response-flow}
\end{figure}
\FloatBarrier

\section{Scope and Assumptions}
\begin{itemize}
\item Supported input is a URL in request payload. Scan modes include quick, deep, and custom paths.
\item Out of scope: full browser sandboxing guarantees, user education workflow, and policy enforcement outside this API tool.
\item Trust boundaries: user input is untrusted; external intelligence services are semi-trusted; internal scoring logic is trusted system code.
\end{itemize}

\section{Inputs and Outputs}
\textbf{Inputs}
\begin{itemize}
\item URL field: required, non-empty.
\item Scheme expected: HTTP/HTTPS.
\item URL normalization details: Not evidenced in the reviewed codebase.
\item Size and payload limits for request body: Not evidenced in the reviewed codebase.
\item Custom scan option accepts selected checks list.
\end{itemize}

\textbf{Outputs}
\begin{itemize}
\item Risk object with score, level, and reasons.
\item Check-level details (domain/reputation/redirect/SSL and optional deep checks).
\item Indicator flags and metadata fields: exact schema Not evidenced in the reviewed codebase.
\end{itemize}

\begin{figure}[H]
  \centering
  \includegraphics[width=\linewidth]{figures/ui-link-analyzer-input.png}
  \caption{Link Analyzer user interface (URL input and scan modes). UI label reflects an earlier prototype name; in this report it is referred to as Link Analyzer.}
  \label{fig:link-analyzer-ui-input}
\end{figure}
\FloatBarrier

\begin{figure}[H]
  \centering
  \includegraphics[width=\linewidth]{figures/ui-link-analyzer-custom-scan-options.png}
  \caption{Link Analyzer custom scan configuration interface with selectable inspection options.}
  \label{fig:link-analyzer-ui-custom-scan}
\end{figure}
\FloatBarrier

\begin{figure}[H]
  \centering
  \includegraphics[width=\linewidth]{figures/ui-link-analyzer-result-low.png}
  \caption{Link Analyzer result interface showing a low-risk classification.}
  \label{fig:link-analyzer-ui-low}
\end{figure}
\FloatBarrier

\begin{figure}[H]
  \centering
  \includegraphics[width=\linewidth]{figures/ui-link-analyzer-result-review.png}
  \caption{Link Analyzer result interface showing a needs-review classification.}
  \label{fig:link-analyzer-ui-review}
\end{figure}
\FloatBarrier

\begin{figure}[H]
  \centering
  \includegraphics[width=\linewidth]{figures/ui-link-analyzer-result-high.png}
  \caption{Link Analyzer result interface showing a high-risk classification.}
  \label{fig:link-analyzer-ui-high}
\end{figure}
\FloatBarrier

\begin{figure}[H]
  \centering
  \includegraphics[width=\linewidth]{figures/fig-backend-processing-pipeline.png}
  \caption{Link Analyzer: Redirect and Reputation Pipeline}
  \label{fig:link-analyzer-redirect-reputation-pipeline}
\end{figure}
\FloatBarrier

\begin{figure}[H]
  \centering
  \includegraphics[width=\linewidth]{figures/fig-system-architecture-overview.png}
  \caption{Link Analyzer: External intelligence inputs and resulting tool output indicators in the system architecture context.}
  \label{fig:link-analyzer-threat-indicators-output}
\end{figure}
\FloatBarrier

\section{Processing Pipeline}
\begin{enumerate}
\item Accept and validate URL request.
\item Apply target safety checks before outbound processing.
\item Execute selected checks (quick, deep, or custom path).
\item Collect check outputs and compute aggregate risk.
\item Return structured response and record tool usage.
\end{enumerate}

\textbf{Error handling behavior}
\begin{itemize}
\item Invalid input is rejected before analysis.
\item External API timeout or failure handling policy: Not evidenced in the reviewed codebase.
\item Missing data from one check is handled conservatively in scoring.
\end{itemize}

\textbf{Logging and audit}
\begin{itemize}
\item History entries and tool counters are updated per scan.
\item Sensitive payload retention policy details: Not evidenced in the reviewed codebase.
\item Authentication secrets are not expected in this tool input.
\end{itemize}

\section{Detection Logic}
\begin{itemize}
\item Indicators include domain age, redirect behavior, SSL status, IP/geolocation context, and reputation signals.
\item Optional deep checks include dynamic behavior, favicon signals, and brand mismatch indicators.
\item Score combination logic is rule-based/weighted in practice, but exact weights and thresholds are Not evidenced in the reviewed codebase.
\end{itemize}

\section{Security and Abuse Resistance}
\begin{itemize}
\item Input is validated and normalized before processing.
\item URL target filtering is used to reduce SSRF exposure, including private/loopback target rejection.
\item Rate limiting and abuse controls at gateway/API level: Not evidenced in the reviewed codebase.
\item Tool usage counters support abuse monitoring.
\end{itemize}

\section{Privacy Considerations}
\begin{itemize}
\item Tool accepts URL input and returns analysis output.
\item Stored versus transient fields in history are deployment-dependent: Not evidenced in the reviewed codebase.
\item Token handling relies on platform authentication layer, not this tool-specific logic.
\end{itemize}

\section{Limitations}
\begin{itemize}
\item Results depend on external service availability and quality.
\item Dynamic checks may not cover all evasive behavior.
\item SSRF protections are hostname/target-rule based and require strict maintenance.
\item Risk score is an analyst aid, not proof of compromise.
\end{itemize}

\section{Test Plan (Scalable)}
\textbf{Unit tests}
\begin{itemize}
\item URL parser and validator edge cases.
\item Score calculation for isolated indicator combinations.
\item Check selector behavior for custom scan mode.
\end{itemize}

\textbf{Integration tests}
\begin{itemize}
\item End-to-end API response structure for each scan mode.
\item External API success/failure fallback behavior.
\end{itemize}

\textbf{Adversarial tests}
\begin{itemize}
\item Redirect chains, homograph-like domains, and suspicious TLS setups.
\item Known phishing URLs and benign controls.
\end{itemize}

\textbf{Performance tests}
\begin{itemize}
\item Latency by scan mode.
\item Timeout behavior under upstream API delays.
\end{itemize}

\section{Operational Notes}
\begin{itemize}
\item Monitor scan failures, timeout rates, and risk-level distributions.
\item Monitor tool counter growth for usage and abuse patterns.
\item Troubleshooting steps: verify API keys, check upstream service status, inspect validation failures, and review history entries.
\end{itemize}

\subsection{Audit Trail and History Logging}
\begin{figure}[H]
  \centering
  \includegraphics[width=\linewidth]{figures/history-link-analyzer-filter.png}
  \caption{Audit trail showing historical Link Analyzer executions filtered by tool.}
  \label{fig:history-link-analyzer}
\end{figure}
\FloatBarrier
