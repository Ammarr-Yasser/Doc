\documentclass[12pt,a4paper]{report}

% ===== Page layout =====
\usepackage[a4paper,margin=2.2cm]{geometry}
\raggedbottom

% ===== Fonts and typography =====
\usepackage[T1]{fontenc}
\usepackage[utf8]{inputenc}
\usepackage{lmodern}
\usepackage{microtype}
\usepackage{seqsplit}
\usepackage{setspace}
\onehalfspacing

% ===== Line breaking and overflow mitigation =====
\setlength{\emergencystretch}{3em}
\tolerance=2000
\pretolerance=1000
\hyphenpenalty=200
\exhyphenpenalty=200
\sloppy
\newcommand{\codewrap}[1]{\texttt{\seqsplit{#1}}}

% ===== Figures, tables, and floats =====
\usepackage{graphicx}
\usepackage{float}
\usepackage{longtable}
\usepackage{booktabs}
\usepackage{array}
\usepackage{placeins}

% ===== Code listings (no shell-escape required) =====
\usepackage{listings}
\usepackage{xcolor}
\usepackage{framed}
\usepackage{fancyvrb}
\lstset{
  basicstyle=\ttfamily\small,
  breaklines=true,
  breakatwhitespace=true,
  frame=single,
  columns=fullflexible,
  keepspaces=true,
  showstringspaces=false
}

% ===== Links and PDF navigation =====
\usepackage{xurl}
\usepackage[hidelinks,hypertexnames=false]{hyperref}
\usepackage{bookmark}
\Urlmuskip=0mu plus 1mu

% ===== Headers/footers =====
\usepackage{fancyhdr}
\pagestyle{fancy}
\fancyhf{}
\fancyhead[L]{DefendX Graduation Project}
\fancyhead[R]{\leftmark}
\fancyfoot[C]{\thepage}
\setlength{\headheight}{15pt}
\addtolength{\topmargin}{-3pt}

% Pandoc-style list helper used by generated section files.
\providecommand{\tightlist}{}

% Pandoc syntax-highlighting compatibility for generated section files.
\definecolor{shadecolor}{RGB}{248,248,248}
\newenvironment{Shaded}{\begin{snugshade}}{\end{snugshade}}
\DefineVerbatimEnvironment{Highlighting}{Verbatim}{commandchars=\\\{\},fontsize=\small}

% Pandoc token macros (safe no-op defaults).
\providecommand{\NormalTok}[1]{#1}
\providecommand{\FunctionTok}[1]{#1}
\providecommand{\DataTypeTok}[1]{#1}
\providecommand{\StringTok}[1]{#1}
\providecommand{\DecValTok}[1]{#1}
\providecommand{\OperatorTok}[1]{#1}
\providecommand{\CommentTok}[1]{#1}
\providecommand{\KeywordTok}[1]{#1}
\providecommand{\BuiltInTok}[1]{#1}
\providecommand{\VariableTok}[1]{#1}
\providecommand{\ConstantTok}[1]{#1}
\providecommand{\CharTok}[1]{#1}
\providecommand{\SpecialCharTok}[1]{#1}
\providecommand{\FloatTok}[1]{#1}
\providecommand{\AlertTok}[1]{#1}
\providecommand{\InformationTok}[1]{#1}
\providecommand{\WarningTok}[1]{#1}
\providecommand{\ErrorTok}[1]{#1}
\providecommand{\OtherTok}[1]{#1}
\providecommand{\RegionMarkerTok}[1]{#1}
\providecommand{\AnnotationTok}[1]{#1}
\providecommand{\AttributeTok}[1]{#1}
\providecommand{\ControlFlowTok}[1]{#1}
\providecommand{\ExtensionTok}[1]{#1}
\providecommand{\ImportTok}[1]{#1}
\providecommand{\PreprocessorTok}[1]{#1}
\providecommand{\DocumentationTok}[1]{#1}

\begin{document}

% =====================
% Front matter (Roman numerals)
% =====================
\pagenumbering{roman}

\begin{titlepage}
  \centering
  \vspace*{1.5cm}

  {\Large \textbf{Helwan University}\par}
  \vspace{0.4cm}
  {\large Faculty of Engineering\par}
  {\large Communication and Electronics Engineering Department\par}

  \vspace{1.5cm}
  {\LARGE \textbf{DefendX}\par}
  \vspace{0.3cm}
  {\large A Web-Based Security Analysis Platform\par}

  \vspace{1.4cm}
  {\large Graduation Project\par}

  \vspace{1.0cm}
  \begin{tabular}{ll}
    \textbf{Prepared by:} & \begin{tabular}[t]{@{}l@{}}
      Ahmed Bakr \\
      Ahmed Ammar \\
      Yasser Mohamed Ragab \\
      Mohamed Sama \\
      Mohamed Shahd \\
      Wael Shorouq \\
      Mohammed Farouq
    \end{tabular} \\
    \\
    \textbf{Supervised by:} & Dr. Ahmed Abdelhaleem \\
    \textbf{Academic Year:} & 2025--2026 \\
    \textbf{Submission Date:} & \underline{\hspace{4cm}} 2026 \\
  \end{tabular}

  \vfill
\end{titlepage}

\chapter*{Abstract}
\addcontentsline{toc}{chapter}{Abstract}
This thesis presents DefendX, a cybersecurity platform that focuses on social engineering risks. The project is implemented as a web system with a frontend, backend services, and external intelligence providers. The technical scope of this report is limited to two core tools: the Link Analyzer and the File Download Checker. The document explains architecture, security controls, threat analysis, and current implementation details for both tools.\par

\chapter*{Acknowledgements}
\addcontentsline{toc}{chapter}{Acknowledgements}
We would like to express our sincere gratitude to our supervisor, Dr. Ahmed Abdelhaleem, for his guidance, constructive feedback, and continuous support throughout this graduation project. We also thank the Communication and Electronics Engineering Department and the Faculty of Engineering at Helwan University for providing the academic environment and resources required to complete this work. This project was achieved through effective collaboration among all team members, with shared responsibility across analysis, implementation, and documentation tasks. We appreciate the professional and technical support that helped us maintain quality, consistency, and discipline during development. We are grateful for the opportunity to complete this work as part of our undergraduate training.\par

\chapter*{List of Abbreviations}
\addcontentsline{toc}{chapter}{List of Abbreviations}
\begin{longtable}{@{}p{0.25\textwidth}p{0.7\textwidth}@{}}
\toprule
\textbf{Acronym} & \textbf{Meaning} \\
\midrule
API & Application Programming Interface \\
JWT & JSON Web Token \\
SSRF & Server-Side Request Forgery \\
TLS & Transport Layer Security \\
\bottomrule
\end{longtable}

\tableofcontents
\listoffigures
\listoftables

\clearpage
\pagenumbering{arabic}

\chapter{Introduction}
\hypertarget{project-overview}{%
\subsection{Project Overview}\label{sec:00-Overview__project-overview:project-overview}}

DefendX is a web-based security analysis platform focused on protecting users from social engineering attacks through a Flask API and a React/Vite dashboard. The system provides the URL Risk Analysis Tool (Link Analyzer), file download scanning, and operational logging.

\hypertarget{purpose}{%
\subsubsection{Purpose}\label{sec:00-Overview__project-overview:purpose}}

DefendX is a security analysis platform designed to identify and assess risks introduced through explicit user inputs, including:

\begin{itemize}
\tightlist
\item
  URLs
\item
  Files
\end{itemize}

The platform provides explainable risk assessments, auditability, and actionable guidance without executing untrusted content.
All analysis tools in DefendX operate exclusively on explicit user input and do not perform background monitoring, credential harvesting, or passive data collection.

\begin{center}\rule{0.5\linewidth}{0.5pt}\end{center}

\hypertarget{core-objectives}{%
\subsubsection{Core Objectives}\label{sec:00-Overview__project-overview:core-objectives}}

DefendX is designed to:

\begin{itemize}
\tightlist
\item
  Detect common and advanced threat indicators
\item
  Provide transparent and explainable risk scores
\item
  Preserve user privacy
\item
  Maintain strong auditability
\item
  Operate safely under partial failure
\end{itemize}

\begin{center}\rule{0.5\linewidth}{0.5pt}\end{center}

\hypertarget{what-defendx-is}{%
\subsubsection{What DefendX Is}\label{sec:00-Overview__project-overview:what-defendx-is}}

\begin{itemize}
\tightlist
\item
  A centralized security analysis engine
\item
  A backend-driven trust decision system
\item
  A modular tool-based architecture
\end{itemize}

\begin{center}\rule{0.5\linewidth}{0.5pt}\end{center}

\hypertarget{what-defendx-is-not}{%
\subsubsection{What DefendX Is Not}\label{sec:00-Overview__project-overview:what-defendx-is-not}}

\begin{itemize}
\tightlist
\item
  An antivirus engine
\item
  A sandbox execution platform (by default)
\item
  A replacement for endpoint security
\item
  A real-time intrusion prevention system
\end{itemize}

\begin{center}\rule{0.5\linewidth}{0.5pt}\end{center}

\hypertarget{security-philosophy}{%
\subsubsection{Security Philosophy}\label{sec:00-Overview__project-overview:security-philosophy}}

DefendX follows these principles:

\begin{itemize}
\tightlist
\item
  Assume all external input is malicious
\item
  Treat the frontend as untrusted
\item
  Prefer false positives over false negatives
\item
  Fail safely and visibly
\item
  Avoid black-box decisions
\end{itemize}

\begin{center}\rule{0.5\linewidth}{0.5pt}\end{center}

\hypertarget{high-level-capabilities}{%
\subsubsection{High-Level Capabilities}\label{sec:00-Overview__project-overview:high-level-capabilities}}

\begin{longtable}[]{@{}ll@{}}
\toprule
Capability & Description \\
\midrule
\endhead
URL Risk Analysis & URL reputation, redirects, SSL, and optional dynamic checks \\
File Analysis & Static file inspection and reputation \\
\bottomrule
\end{longtable}

\begin{center}\rule{0.5\linewidth}{0.5pt}\end{center}

\hypertarget{trust-model}{%
\subsubsection{Trust Model}\label{sec:00-Overview__project-overview:trust-model}}

\begin{itemize}
\tightlist
\item
  Backend: trusted
\item
  Frontend: untrusted
\item
  External services: untrusted but useful
\item
  User input: hostile by default
\end{itemize}

\begin{center}\rule{0.5\linewidth}{0.5pt}\end{center}

\hypertarget{system-boundary}{%
\subsubsection{System Boundary}\label{sec:00-Overview__project-overview:system-boundary}}

DefendX makes security decisions server-side and returns results to clients. It does not execute untrusted content or take autonomous remediation actions.

\hypertarget{scope}{%
\subsubsection{Scope}\label{sec:00-Overview__project-overview:scope}}

\begin{itemize}
\tightlist
\item
  Backend API serving authenticated tool endpoints under \texttt{/api}.
\item
  Frontend SPA that drives authenticated workflows and reports.
\item
  User-invoked URL and file analysis workflows through authenticated API requests.
\end{itemize}

\hypertarget{primary-data-flows}{%
\subsubsection{Primary Data Flows}\label{sec:00-Overview__project-overview:primary-data-flows}}

\begin{itemize}
\tightlist
\item
  Users authenticate via \texttt{/api/auth/login} with RSA-encrypted credential fields.
\item
  Backend issues JWT access and refresh tokens and stores refresh token state in the database.
\item
  Tool endpoints process user inputs, compute risk scores, and log usage in \texttt{History} and \texttt{Tool\_Counter} tables.
\item
  Frontend displays findings, status, and usage history from the API.
\end{itemize}

\hypertarget{data-classes}{%
\subsubsection{Data Classes}\label{sec:00-Overview__project-overview:data-classes}}

\begin{itemize}
\tightlist
\item
  Account data: \texttt{User} (username, email, role, suspended flags).
\item
  Auth data: JWT access and refresh tokens, refresh token server-side records.
\item
  Tool data: user-provided URLs, file metadata, and tool outputs.
\item
  Logs: tool usage history and per-user counters.
\end{itemize}

\hypertarget{external-dependencies}{%
\subsubsection{External Dependencies}\label{sec:00-Overview__project-overview:external-dependencies}}

\begin{itemize}
\tightlist
\item
  VirusTotal, Google Safe Browsing, and IPGeolocation (API-backed checks).
\item
  Playwright for optional dynamic link analysis.
\end{itemize}

\hypertarget{out-of-scope}{%
\subsubsection{Out of Scope}\label{sec:00-Overview__project-overview:out-of-scope}}

\begin{itemize}
\tightlist
\item
  Backend APIs for tools present only in frontend routes (USB, PCAP, Audio, User Behavior) are not confirmed in this repository. See \texttt{00-Overview/known-limitations.md}.
\end{itemize}

\FloatBarrier

\chapter{System Architecture}
\hypertarget{architecture-summary}{%
\subsection{Architecture Summary}\label{sec:00-Overview__architecture-summary:architecture-summary}}

\hypertarget{system-overview}{%
\subsubsection{System Overview}\label{sec:00-Overview__architecture-summary:system-overview}}

DefendX uses a layered architecture with strict separation of concerns:

\begin{itemize}
\tightlist
\item
  Presentation layer (Frontend)
\item
  Security decision layer (Backend)
\item
  External intelligence providers
\item
  Persistent storage
\end{itemize}

\hypertarget{high-level-components}{%
\subsubsection{High-Level Components}\label{sec:00-Overview__architecture-summary:high-level-components}}

\begin{itemize}
\tightlist
\item
  Backend API: Flask + Flask-RESTX with a single API prefix \texttt{/api} and docs at \texttt{/api/docs}.
\item
  Frontend: React/Vite SPA proxied to \texttt{/api} during development and optionally served from \texttt{frontend/dist} or \texttt{frontend/build} in production.
\item
  Data Store: SQLAlchemy models backed by SQLite for \texttt{dev.db} and \texttt{test.db} (production database is TBD).
\item
  Async Tasks: Celery worker factory wired to Flask config (\texttt{CELERY\_BROKER\_URL}, \texttt{CELERY\_RESULT\_BACKEND}).
\end{itemize}

\hypertarget{frontend}{%
\paragraph{Frontend}\label{sec:00-Overview__architecture-summary:frontend}}

\begin{itemize}
\tightlist
\item
  Single-page application
\item
  Handles user interaction
\item
  No security authority
\end{itemize}

\hypertarget{backend}{%
\paragraph{Backend}\label{sec:00-Overview__architecture-summary:backend}}

\begin{itemize}
\tightlist
\item
  Flask-based API
\item
  Central security engine
\item
  Authentication and authorization
\item
  Tool orchestration
\item
  Risk scoring
\end{itemize}

\hypertarget{database}{%
\paragraph{Database}\label{sec:00-Overview__architecture-summary:database}}

\begin{itemize}
\tightlist
\item
  Stores users
\item
  Stores history (audit logs)
\item
  Stores tool usage counters
\end{itemize}

\hypertarget{external-services}{%
\paragraph{External Services}\label{sec:00-Overview__architecture-summary:external-services}}

\begin{itemize}
\tightlist
\item
  Reputation APIs
\item
  Intelligence feeds
\end{itemize}

\hypertarget{backend-entry-points}{%
\subsubsection{Backend Entry Points}\label{sec:00-Overview__architecture-summary:backend-entry-points}}

\begin{itemize}
\tightlist
\item
  \texttt{backend/app/main.py} creates the Flask app via \texttt{create\_app(devconfig)} and registers all namespaces.
\item
  \texttt{backend/app/\_\_init\_\_.py} performs request hardening and serves the SPA when a static build is present.
\end{itemize}

\hypertarget{api-namespaces}{%
\subsubsection{API Namespaces}\label{sec:00-Overview__architecture-summary:api-namespaces}}

\begin{itemize}
\tightlist
\item
  \texttt{auth}: signup, login, refresh, logout, public-key.
\item
  \texttt{user}: admin CRUD and user detail endpoints.
\item
  \texttt{history}: admin and user tool history.
\item
  \texttt{tool\_counter}: per-user and global usage counters.
\item
  \texttt{tools/link-analyzer}: quick, deep, and custom scans.
\item
  \texttt{tools/file-checker}: file upload scanning.
\item
  \texttt{celery}: start and status for a demo task.
\end{itemize}

\hypertarget{trust-boundaries}{%
\subsubsection{Trust Boundaries}\label{sec:00-Overview__architecture-summary:trust-boundaries}}

\begin{itemize}
\tightlist
\item
  Browser clients communicate with the API via JWT-protected endpoints.
\item
  Tool analyzers call out to external reputational APIs and services.
\end{itemize}

\hypertarget{static-web-serving}{%
\subsubsection{Static Web Serving}\label{sec:00-Overview__architecture-summary:static-web-serving}}

\begin{itemize}
\tightlist
\item
  Frontend build artifacts are served by Flask if \texttt{frontend/dist} or \texttt{frontend/build} exists.
\item
  Request hardening blocks path traversal and source file access (e.g., \texttt{.env}, \texttt{.py}, \texttt{.jsx}).
\end{itemize}

\hypertarget{data-flow}{%
\subsubsection{Data Flow}\label{sec:00-Overview__architecture-summary:data-flow}}

\begin{enumerate}
\def\labelenumi{\arabic{enumi}.}
\tightlist
\item
  User submits input via frontend
\item
  Frontend sends request to backend API
\item
  Backend validates authentication and input
\item
  Security tool executes
\item
  Risk score computed
\item
  History and counters updated
\item
  Result returned to frontend
\end{enumerate}

\hypertarget{design-constraints}{%
\subsubsection{Design Constraints}\label{sec:00-Overview__architecture-summary:design-constraints}}

\begin{itemize}
\tightlist
\item
  No direct frontend access to databases
\item
  No frontend access to external intelligence services
\item
  All trust decisions server-side
\item
  Minimal data retention
\end{itemize}

\hypertarget{architectural-benefits}{%
\subsubsection{Architectural Benefits}\label{sec:00-Overview__architecture-summary:architectural-benefits}}

\begin{itemize}
\tightlist
\item
  Strong auditability
\item
  Clear trust boundaries
\item
  Easier security review
\item
  Modular extensibility
\end{itemize}

\FloatBarrier
\hypertarget{architecture-diagrams-plain-text}{%
\subsection{Architecture Diagrams (Plain Text)}\label{sec:00-Overview__architecture-diagrams:architecture-diagrams-plain-text}}

\hypertarget{system-context}{%
\subsubsection{System Context}\label{sec:00-Overview__architecture-diagrams:system-context}}

{[}Browser SPA{]} -\/-\/-\/-HTTPS-\/-\/-\/-\textgreater{} {[}Flask API /api{]}
{[}Flask API /api{]} -\/-\/-\/-\/-\textgreater{} {[}SQLite DB: User, History, Tool\_Counter, RefreshToken{]}
{[}Flask API /api{]} -\/-\/-\/-\/-\textgreater{} {[}External APIs: VirusTotal, Google Safe Browsing, IPGeolocation{]}

\hypertarget{tool-processing-flow}{%
\subsubsection{Tool Processing Flow}\label{sec:00-Overview__architecture-diagrams:tool-processing-flow}}

\begin{enumerate}
\def\labelenumi{\arabic{enumi})}
\tightlist
\item
  Client obtains public RSA key from \texttt{/api/auth/public-key}.
\item
  Client encrypts credentials and calls \texttt{/api/auth/signup} or \texttt{/api/auth/login}.
\item
  Backend decrypts fields, validates credentials, and returns access/refresh tokens.
\item
  Client submits a URL or file to a tool endpoint with an access token.
\item
  Backend runs analysis, calculates risk score, then logs \texttt{History} and updates \texttt{Tool\_Counter}.
\item
  Client renders result, history, and counter metrics.
\end{enumerate}

\hypertarget{optional-dynamic-scan-link-analyzer}{%
\subsubsection{Optional Dynamic Scan (Link Analyzer)}\label{sec:00-Overview__architecture-diagrams:optional-dynamic-scan-link-analyzer}}

User-invoked deep/custom URL scans may use Playwright to collect dynamic page indicators. If Playwright is not installed or disabled, this step may fail and should be treated as optional or degraded.

\hypertarget{architecture-diagrams-text-description}{%
\subsection{Architecture Diagrams (Text Description)}\label{sec:00-Overview__architecture-diagrams:architecture-diagrams-text-description}}

\hypertarget{system-architecture}{%
\subsubsection{System Architecture}\label{sec:00-Overview__architecture-diagrams:system-architecture}}

User → Frontend SPA → DefendX Backend → Security Tools → Database

The frontend communicates with the backend exclusively via HTTPS. All authentication and authorization occur in the backend.

\begin{center}\rule{0.5\linewidth}{0.5pt}\end{center}

\hypertarget{tool-execution-flow}{%
\subsubsection{Tool Execution Flow}\label{sec:00-Overview__architecture-diagrams:tool-execution-flow}}

Request
→ JWT validation\\
→ Input validation\\
→ Tool execution\\
→ Risk scoring\\
→ History logging\\
→ Response

\begin{center}\rule{0.5\linewidth}{0.5pt}\end{center}

\hypertarget{external-services-interaction}{%
\subsubsection{External Services Interaction}\label{sec:00-Overview__architecture-diagrams:external-services-interaction}}

The backend communicates with:

\begin{itemize}
\tightlist
\item
  VirusTotal
\item
  Google Safe Browsing
\item
  IP intelligence providers
\item
  Additional reputation or URL intelligence providers (deployment-dependent)
\end{itemize}

The frontend never communicates directly with these services.

\begin{center}\rule{0.5\linewidth}{0.5pt}\end{center}

\hypertarget{authentication-flow}{%
\subsubsection{Authentication Flow}\label{sec:00-Overview__architecture-diagrams:authentication-flow}}

\begin{enumerate}
\def\labelenumi{\arabic{enumi}.}
\tightlist
\item
  User logs in
\item
  Backend issues JWT access and refresh tokens
\item
  Frontend uses access token for API calls
\item
  Backend validates token on each request
\end{enumerate}

\begin{center}\rule{0.5\linewidth}{0.5pt}\end{center}

\hypertarget{security-boundary}{%
\subsubsection{Security Boundary}\label{sec:00-Overview__architecture-diagrams:security-boundary}}

The backend is the sole trusted component. Compromise of the frontend does not imply compromise of the security engine.

\FloatBarrier
\hypertarget{frontend-architecture}{%
\subsection{Frontend Architecture}\label{sec:04-Frontend__frontend-architecture:frontend-architecture}}

\hypertarget{framework}{%
\subsubsection{Framework}\label{sec:04-Frontend__frontend-architecture:framework}}

\begin{itemize}
\tightlist
\item
  React SPA built with Vite.
\item
  Routing is managed with React Router (\texttt{frontend/src/router/AppRouter.jsx}).
\end{itemize}

\hypertarget{app-layout-and-routing}{%
\subsubsection{App Layout and Routing}\label{sec:04-Frontend__frontend-architecture:app-layout-and-routing}}

\begin{itemize}
\tightlist
\item
  Public routes: \texttt{/} (login), \texttt{/register}.
\item
  Protected routes: \texttt{/home}, \texttt{/profile}, \texttt{/history}, tool pages, and documentation pages.
\item
  Admin routes: \texttt{/admins} (enforced via \texttt{AdminRoute}).
\end{itemize}

\hypertarget{api-client}{%
\subsubsection{API Client}\label{sec:04-Frontend__frontend-architecture:api-client}}

\begin{itemize}
\tightlist
\item
  Axios instance configured with base URL from \texttt{VITE\_API\_URL} or \texttt{/api} by default.
\item
  Request middleware attaches access token from \texttt{localStorage}.
\item
  Response middleware refreshes access token via \texttt{/api/auth/refresh} and retries once.
\end{itemize}

\hypertarget{tool-pages}{%
\subsubsection{Tool Pages}\label{sec:04-Frontend__frontend-architecture:tool-pages}}

\begin{itemize}
\tightlist
\item
  Link Analyzer and File Download Checker.
\item
  Additional tool pages exist (USB, PCAP, Audio, Behavioral Risk) but backend APIs are not confirmed in this repo. Not evidenced in the reviewed codebase.
\end{itemize}

\hypertarget{state-and-storage}{%
\subsubsection{State and Storage}\label{sec:04-Frontend__frontend-architecture:state-and-storage}}

\begin{itemize}
\tightlist
\item
  Tokens and user profile are stored in \texttt{localStorage}.
\item
  Admin role checks are derived from locally stored user payload.
\end{itemize}

\hypertarget{code-alignment-notes}{%
\subsubsection{Code Alignment Notes}\label{sec:04-Frontend__frontend-architecture:code-alignment-notes}}

\begin{itemize}
\tightlist
\item
  \texttt{/tools/user-behavior/analyze} is referenced in \texttt{frontend/src/api/index.js}, but no backend endpoint was found. Not evidenced in the reviewed codebase.
\end{itemize}

\hypertarget{purpose}{%
\subsubsection{Purpose}\label{sec:04-Frontend__frontend-architecture:purpose}}

The DefendX frontend is the \textbf{user-facing interaction layer} of the platform. Its role is to collect user input, invoke backend APIs, and present security results in a clear and actionable manner.

The frontend does \textbf{not} make security decisions and is treated as an untrusted component.

\begin{center}\rule{0.5\linewidth}{0.5pt}\end{center}

\hypertarget{architectural-role}{%
\subsubsection{Architectural Role}\label{sec:04-Frontend__frontend-architecture:architectural-role}}

Within the overall system, the frontend acts as:

\begin{itemize}
\tightlist
\item
  A presentation layer
\item
  A request orchestrator
\item
  A visualization and UX layer
\end{itemize}

All security logic, validation, and risk assessment occur in the backend.

\begin{center}\rule{0.5\linewidth}{0.5pt}\end{center}

\hypertarget{high-level-structure}{%
\subsubsection{High-Level Structure}\label{sec:04-Frontend__frontend-architecture:high-level-structure}}

The frontend is implemented as a \textbf{single-page application (SPA)} with:

\begin{itemize}
\tightlist
\item
  Component-based UI
\item
  Client-side routing
\item
  Centralized API client
\item
  Token-aware request handling
\end{itemize}

It communicates with the backend exclusively over HTTPS.

\begin{center}\rule{0.5\linewidth}{0.5pt}\end{center}

\hypertarget{api-communication-model}{%
\subsubsection{API Communication Model}\label{sec:04-Frontend__frontend-architecture:api-communication-model}}

\hypertarget{centralized-api-client}{%
\paragraph{Centralized API Client}\label{sec:04-Frontend__frontend-architecture:centralized-api-client}}

The frontend uses a centralized API client responsible for:

\begin{itemize}
\tightlist
\item
  Attaching the \texttt{Authorization:\ Bearer\ \textless{}access\_token\textgreater{}} header
\item
  Normalizing error handling
\item
  Triggering token refresh logic when needed
\item
  Retrying failed requests after refresh
\end{itemize}

This prevents duplicated authentication logic across components.

\begin{center}\rule{0.5\linewidth}{0.5pt}\end{center}

\hypertarget{authentication-flow-client-perspective}{%
\subsubsection{Authentication Flow (Client Perspective)}\label{sec:04-Frontend__frontend-architecture:authentication-flow-client-perspective}}

\begin{enumerate}
\def\labelenumi{\arabic{enumi}.}
\tightlist
\item
  User submits login credentials
\item
  Frontend sends credentials to backend authentication endpoint
\item
  Backend returns access and refresh tokens
\item
  Frontend stores access token and begins authenticated requests
\end{enumerate}

The frontend treats authentication failures as session state changes.

\begin{center}\rule{0.5\linewidth}{0.5pt}\end{center}

\hypertarget{tool-interaction-flows}{%
\subsubsection{Tool Interaction Flows}\label{sec:04-Frontend__frontend-architecture:tool-interaction-flows}}

\hypertarget{link-analyzer}{%
\paragraph{Link Analyzer}\label{sec:04-Frontend__frontend-architecture:link-analyzer}}

\begin{enumerate}
\def\labelenumi{\arabic{enumi}.}
\tightlist
\item
  User enters a URL
\item
  Frontend performs basic format validation
\item
  Backend Quick Scan is invoked
\item
  Results are rendered
\item
  User may optionally trigger Deep or Custom scans
\end{enumerate}

\begin{center}\rule{0.5\linewidth}{0.5pt}\end{center}

\hypertarget{file-download-checker}{%
\paragraph{File Download Checker}\label{sec:04-Frontend__frontend-architecture:file-download-checker}}

\begin{enumerate}
\def\labelenumi{\arabic{enumi}.}
\tightlist
\item
  User uploads a file
\item
  Frontend checks file size limits
\item
  File sent to backend
\item
  Backend returns hash and risk verdict
\item
  Frontend clearly labels unsafe files
\end{enumerate}

\begin{center}\rule{0.5\linewidth}{0.5pt}\end{center}

\hypertarget{history-view}{%
\subsubsection{History View}\label{sec:04-Frontend__frontend-architecture:history-view}}

The frontend retrieves history entries to display:

\begin{itemize}
\tightlist
\item
  Tool used
\item
  Timestamp
\item
  Risk outcome
\end{itemize}

This allows users to review past security checks.

\begin{center}\rule{0.5\linewidth}{0.5pt}\end{center}

\hypertarget{error-handling-strategy}{%
\subsubsection{Error Handling Strategy}\label{sec:04-Frontend__frontend-architecture:error-handling-strategy}}

The frontend categorizes errors as:

\begin{itemize}
\tightlist
\item
  Validation errors (user-correctable)
\item
  Authentication errors (session-related)
\item
  Tool failures (external dependency issues)
\item
  System errors (unexpected)
\end{itemize}

Errors are shown explicitly without hiding risk.

\begin{center}\rule{0.5\linewidth}{0.5pt}\end{center}

\hypertarget{build-and-serve-models}{%
\subsubsection{Build and Serve Models}\label{sec:04-Frontend__frontend-architecture:build-and-serve-models}}

Two supported deployment models:

\hypertarget{separate-deployment}{%
\paragraph{Separate Deployment}\label{sec:04-Frontend__frontend-architecture:separate-deployment}}

\begin{itemize}
\tightlist
\item
  Frontend hosted independently
\item
  Backend exposed as API service
\end{itemize}

\hypertarget{backend-served-static-assets}{%
\paragraph{Backend-Served Static Assets}\label{sec:04-Frontend__frontend-architecture:backend-served-static-assets}}

\begin{itemize}
\tightlist
\item
  Frontend built into static files
\item
  Backend serves SPA with fallback routing
\end{itemize}

Both models preserve identical security behavior.

\begin{center}\rule{0.5\linewidth}{0.5pt}\end{center}

\hypertarget{frontend-guarantees}{%
\subsubsection{Frontend Guarantees}\label{sec:04-Frontend__frontend-architecture:frontend-guarantees}}

The frontend guarantees:

\begin{itemize}
\tightlist
\item
  No security decisions
\item
  No secret storage beyond tokens
\item
  No direct external API calls
\item
  Clear representation of backend results
\end{itemize}

\FloatBarrier
\hypertarget{backend-overview}{%
\subsection{Backend Overview}\label{sec:01-Backend__backend-overview:backend-overview}}

\hypertarget{entry-points}{%
\subsubsection{Entry Points}\label{sec:01-Backend__backend-overview:entry-points}}

\begin{itemize}
\tightlist
\item
  \texttt{backend/app/main.py} creates the Flask app via \texttt{create\_app(devconfig)} and registers namespaces under \texttt{/api}.
\item
  The API documentation is served at \texttt{/api/docs} via Flask-RESTX.
\end{itemize}

\hypertarget{app-factory-and-hardening}{%
\subsubsection{App Factory and Hardening}\label{sec:01-Backend__backend-overview:app-factory-and-hardening}}

\begin{itemize}
\tightlist
\item
  \texttt{backend/app/\_\_init\_\_.py} provides \texttt{create\_app()} and request hardening.
\item
  The app blocks traversal markers and denies access to source file patterns such as \texttt{.env}, \texttt{.py}, \texttt{.js}, \texttt{.jsx}.
\item
  If \texttt{frontend/dist} or \texttt{frontend/build} is present, Flask serves the SPA and its assets.
\end{itemize}

\hypertarget{namespaces-and-routes}{%
\subsubsection{Namespaces and Routes}\label{sec:01-Backend__backend-overview:namespaces-and-routes}}

\begin{itemize}
\tightlist
\item
  \texttt{auth}: signup, login, refresh, logout, logout-all, public-key.
\item
  \texttt{user}: admin CRUD on users and self-access read access.
\item
  \texttt{history}: admin history and per-user history.
\item
  \texttt{tool\_counter}: per-user and global usage counters.
\item
  \texttt{tools/link-analyzer}: quick, deep, and custom scans.
\item
  \texttt{tools/file-checker}: file upload scanning.
\item
  \texttt{celery}: start and status for a demo task.
\end{itemize}

\hypertarget{data-access}{%
\subsubsection{Data Access}\label{sec:01-Backend__backend-overview:data-access}}

\begin{itemize}
\tightlist
\item
  SQLAlchemy models in \texttt{backend/app/models.py}.
\item
  Alembic migrations live in \texttt{migrations/}.
\end{itemize}

\hypertarget{code-alignment-notes}{%
\subsubsection{Code Alignment Notes}\label{sec:01-Backend__backend-overview:code-alignment-notes}}

\begin{itemize}
\tightlist
\item
  Some tool routes appear in the frontend without matching backend namespaces (USB, PCAP, Audio, Behavioral Risk). Not evidenced in the reviewed codebase.
\end{itemize}

\hypertarget{backend-overview-1}{%
\subsection{Backend Overview}\label{sec:01-Backend__backend-overview:backend-overview-1}}

\hypertarget{purpose}{%
\subsubsection{Purpose}\label{sec:01-Backend__backend-overview:purpose}}

The DefendX backend is the \textbf{central security decision engine} of the platform. It is responsible for authentication, authorization, tool orchestration, risk scoring, audit logging, and controlled interaction with external intelligence services.

All trust decisions are made exclusively in the backend.

\begin{center}\rule{0.5\linewidth}{0.5pt}\end{center}

\hypertarget{core-responsibilities}{%
\subsubsection{Core Responsibilities}\label{sec:01-Backend__backend-overview:core-responsibilities}}

The backend is responsible for:

\begin{itemize}
\tightlist
\item
  Exposing a RESTful API under \texttt{/api}
\item
  Authenticating users using JWT
\item
  Authorizing privileged actions
\item
  Executing security tools
\item
  Aggregating and scoring risk signals
\item
  Writing audit history
\item
  Tracking tool usage
\item
  Enforcing request-level security controls
\end{itemize}

\begin{center}\rule{0.5\linewidth}{0.5pt}\end{center}

\hypertarget{technology-stack}{%
\subsubsection{Technology Stack}\label{sec:01-Backend__backend-overview:technology-stack}}

\begin{itemize}
\tightlist
\item
  Python
\item
  Flask (application framework)
\item
  Flask-RESTX (API namespaces and Swagger docs)
\item
  JWT-based authentication
\item
  Relational database (SQLite in development)
\end{itemize}

\begin{center}\rule{0.5\linewidth}{0.5pt}\end{center}

\hypertarget{application-entry-point}{%
\subsubsection{Application Entry Point}\label{sec:01-Backend__backend-overview:application-entry-point}}

The backend is started via a main application entry point which:

\begin{enumerate}
\def\labelenumi{\arabic{enumi}.}
\tightlist
\item
  Creates the Flask application using an app factory
\item
  Loads configuration from environment variables
\item
  Registers API namespaces
\item
  Enables API documentation at \texttt{/docs}
\item
  Optionally serves frontend static assets
\end{enumerate}

\begin{center}\rule{0.5\linewidth}{0.5pt}\end{center}

\hypertarget{api-structure}{%
\subsubsection{API Structure}\label{sec:01-Backend__backend-overview:api-structure}}

All API endpoints are mounted under:

/api

Namespaces are used to group functionality logically, such as:

\begin{itemize}
\tightlist
\item
  Authentication
\item
  User management
\item
  Individual security tools
\item
  Infrastructure services (history, counters)
\end{itemize}

This modular approach improves maintainability and auditability.

\begin{center}\rule{0.5\linewidth}{0.5pt}\end{center}

\hypertarget{backend-as-a-trust-boundary}{%
\subsubsection{Backend as a Trust Boundary}\label{sec:01-Backend__backend-overview:backend-as-a-trust-boundary}}

The backend is designed as the \textbf{only trusted component} in the system:

\begin{itemize}
\tightlist
\item
  It is the only component with database access
\item
  It is the only component allowed to call external intelligence APIs
\item
  It is the only component that evaluates risk
\end{itemize}

Compromise of the frontend does not compromise backend security decisions.

\begin{center}\rule{0.5\linewidth}{0.5pt}\end{center}

\hypertarget{error-handling-philosophy}{%
\subsubsection{Error Handling Philosophy}\label{sec:01-Backend__backend-overview:error-handling-philosophy}}

\begin{itemize}
\tightlist
\item
  Authentication and validation errors fail fast
\item
  Tool failures return partial results when possible
\item
  External API failures increase risk instead of masking it
\item
  Internal observability failures do not block tool execution
\end{itemize}

\begin{center}\rule{0.5\linewidth}{0.5pt}\end{center}

\hypertarget{backend-guarantees}{%
\subsubsection{Backend Guarantees}\label{sec:01-Backend__backend-overview:backend-guarantees}}

The backend guarantees:

\begin{itemize}
\tightlist
\item
  No execution of untrusted content
\item
  No silent security decisions
\item
  No implicit trust in client input
\item
  Deterministic and explainable outputs
\end{itemize}

\FloatBarrier

\chapter{Security Model and Threat Analysis}
\hypertarget{threat-model}{%
\subsection{Threat Model}\label{sec:00-Overview__threat-model:threat-model}}

\hypertarget{assets}{%
\subsubsection{Assets}\label{sec:00-Overview__threat-model:assets}}

\begin{itemize}
\tightlist
\item
  User identities, roles, and account status.
\item
  JWT access and refresh tokens.
\item
  URLs and file hashes submitted for inspection.
\item
  Audit and usage logs.
\end{itemize}

\hypertarget{trust-boundaries}{%
\subsubsection{Trust Boundaries}\label{sec:00-Overview__threat-model:trust-boundaries}}

\begin{itemize}
\tightlist
\item
  Browser SPA is an untrusted client.
\item
  Backend API is the primary trust enforcement point.
\item
  External reputation providers are third-party dependencies.
\item
  Database stores authentication and logging state.
\end{itemize}

\hypertarget{stride-analysis}{%
\subsubsection{STRIDE Analysis}\label{sec:00-Overview__threat-model:stride-analysis}}

\hypertarget{spoofing}{%
\paragraph{Spoofing}\label{sec:00-Overview__threat-model:spoofing}}

\begin{itemize}
\tightlist
\item
  Risk: Credential reuse or token theft.
\item
  Mitigation: RSA-encrypted credential fields on the client, JWT access + refresh with rotation, refresh token blocklist enforcement.
\end{itemize}

\hypertarget{tampering}{%
\paragraph{Tampering}\label{sec:00-Overview__threat-model:tampering}}

\begin{itemize}
\tightlist
\item
  Risk: Client-side modification of payloads or headers.
\item
  Mitigation: JWT verification on protected endpoints; request validation via Flask-RESTX models for tool inputs.
\end{itemize}

\hypertarget{repudiation}{%
\paragraph{Repudiation}\label{sec:00-Overview__threat-model:repudiation}}

\begin{itemize}
\tightlist
\item
  Risk: Users deny having run tools or changed account data.
\item
  Mitigation: \texttt{History} and \texttt{Tool\_Counter} logging on tool endpoints.
\end{itemize}

\hypertarget{information-disclosure}{%
\paragraph{Information Disclosure}\label{sec:00-Overview__threat-model:information-disclosure}}

\begin{itemize}
\tightlist
\item
  Risk: Access to source files, \texttt{.env}, or backend code via path traversal.
\item
  Mitigation: Request path hardening and forbidden file marker checks in \texttt{backend/app/\_\_init\_\_.py}.
\end{itemize}

\hypertarget{denial-of-service}{%
\paragraph{Denial of Service}\label{sec:00-Overview__threat-model:denial-of-service}}

\begin{itemize}
\tightlist
\item
  Risk: Excessive scans, file uploads, or dynamic scans.
\item
  Mitigation: None visible in code; rate limiting and quotas are Not evidenced in the reviewed codebase.
\end{itemize}

\hypertarget{elevation-of-privilege}{%
\paragraph{Elevation of Privilege}\label{sec:00-Overview__threat-model:elevation-of-privilege}}

\begin{itemize}
\tightlist
\item
  Risk: Non-admins accessing admin routes.
\item
  Mitigation: \texttt{require\_admin()} checks JWT role claim; admin-only routes enforce role checks.
\end{itemize}

\hypertarget{additional-security-notes}{%
\subsubsection{Additional Security Notes}\label{sec:00-Overview__threat-model:additional-security-notes}}

\begin{itemize}
\tightlist
\item
  Link Analyzer blocks private/loopback IP targets to reduce SSRF risk (\texttt{is\_public\_http}).
\item
  Refresh token revocation is enforced server-side using \texttt{RefreshToken} table.
\item
  Authentication token state is stored server-side with expiry and revocation controls.
\end{itemize}

\hypertarget{open-threats}{%
\subsubsection{Open Threats}\label{sec:00-Overview__threat-model:open-threats}}

\begin{itemize}
\tightlist
\item
  Rate limiting, WAF, and abuse detection are not shown in code (Not evidenced in the reviewed codebase).
\item
  Secure file storage for uploaded artifacts beyond temporary files is Not evidenced in the reviewed codebase.
\item
  Production database hardening and encryption at rest are Not evidenced in the reviewed codebase.
\end{itemize}

\hypertarget{threat-model-stride}{%
\subsection{Threat Model (STRIDE)}\label{sec:00-Overview__threat-model:threat-model-stride}}

\hypertarget{spoofing-1}{%
\subsubsection{Spoofing}\label{sec:00-Overview__threat-model:spoofing-1}}

\begin{itemize}
\tightlist
\item
  Threat: Token impersonation
\item
  Control: JWT signature validation, short token lifetime
\end{itemize}

\hypertarget{tampering-1}{%
\subsubsection{Tampering}\label{sec:00-Overview__threat-model:tampering-1}}

\begin{itemize}
\tightlist
\item
  Threat: Request manipulation
\item
  Control: Server-side validation, immutable history
\end{itemize}

\hypertarget{repudiation-1}{%
\subsubsection{Repudiation}\label{sec:00-Overview__threat-model:repudiation-1}}

\begin{itemize}
\tightlist
\item
  Threat: User denies actions
\item
  Control: Per-user audit logging
\end{itemize}

\hypertarget{information-disclosure-1}{%
\subsubsection{Information Disclosure}\label{sec:00-Overview__threat-model:information-disclosure-1}}

\begin{itemize}
\tightlist
\item
  Threat: Sensitive data disclosure
\item
  Control: Hash-only processing, no raw data storage
\end{itemize}

\hypertarget{denial-of-service-1}{%
\subsubsection{Denial of Service}\label{sec:00-Overview__threat-model:denial-of-service-1}}

\begin{itemize}
\tightlist
\item
  Threat: Resource exhaustion
\item
  Control: File size limits, tool counters
\end{itemize}

\hypertarget{elevation-of-privilege-1}{%
\subsubsection{Elevation of Privilege}\label{sec:00-Overview__threat-model:elevation-of-privilege-1}}

\begin{itemize}
\tightlist
\item
  Threat: Unauthorized admin access
\item
  Control: Role-based access control
\end{itemize}

\begin{center}\rule{0.5\linewidth}{0.5pt}\end{center}

\hypertarget{residual-risks}{%
\subsubsection{Residual Risks}\label{sec:00-Overview__threat-model:residual-risks}}

\begin{itemize}
\tightlist
\item
  SSRF via DNS rebinding
\item
  Reputation API dependency
\item
  Heuristic scoring limitations
\end{itemize}

All residual risks are documented and accepted.

\FloatBarrier
\hypertarget{security-controls-mapping}{%
\subsection{Security Controls Mapping}\label{sec:00-Overview__security-controls-mapping:security-controls-mapping}}

This is a control-to-implementation map based on the current repository. Items not evidenced in code are marked TBD.

\hypertarget{access-control}{%
\subsubsection{Access Control}\label{sec:00-Overview__security-controls-mapping:access-control}}

\begin{itemize}
\tightlist
\item
  Control: Role-based access enforcement for admin operations.
\item
  Evidence: \texttt{backend/app/user.py} uses \texttt{require\_admin()} and role checks; \texttt{backend/app/\_\_init\_\_.py} defines \texttt{require\_admin()}.
\item
  Status: Implemented.
\end{itemize}

\hypertarget{authentication-and-session-management}{%
\subsubsection{Authentication and Session Management}\label{sec:00-Overview__security-controls-mapping:authentication-and-session-management}}

\begin{itemize}
\tightlist
\item
  Control: Password hashing and JWT-based session tokens with refresh rotation.
\item
  Evidence: \texttt{backend/app/auth.py}, \texttt{backend/app/models.py} (\texttt{RefreshToken}).
\item
  Status: Implemented.
\end{itemize}

\hypertarget{credential-transport-security}{%
\subsubsection{Credential Transport Security}\label{sec:00-Overview__security-controls-mapping:credential-transport-security}}

\begin{itemize}
\tightlist
\item
  Control: RSA encryption for credential fields in the client, decryption server-side.
\item
  Evidence: \texttt{frontend/src/api/encryption.js}, \texttt{backend/app/security.py}.
\item
  Status: Implemented.
\end{itemize}

\hypertarget{audit-logging}{%
\subsubsection{Audit Logging}\label{sec:00-Overview__security-controls-mapping:audit-logging}}

\begin{itemize}
\tightlist
\item
  Control: Tool usage history with timestamps and user metadata.
\item
  Evidence: \texttt{backend/app/history.py}, \texttt{backend/app/models.py}.
\item
  Status: Implemented.
\end{itemize}

\hypertarget{operational-metrics}{%
\subsubsection{Operational Metrics}\label{sec:00-Overview__security-controls-mapping:operational-metrics}}

\begin{itemize}
\tightlist
\item
  Control: Per-user counters for tool usage.
\item
  Evidence: \texttt{backend/app/toolcount.py}, \texttt{backend/app/models.py}.
\item
  Status: Implemented.
\end{itemize}

\hypertarget{input-validation-and-hardening}{%
\subsubsection{Input Validation and Hardening}\label{sec:00-Overview__security-controls-mapping:input-validation-and-hardening}}

\begin{itemize}
\tightlist
\item
  Control: Request validation models and URL target filtering.
\item
  Evidence: Flask-RESTX models in \texttt{backend/Tools/link\_analyzer/route/schemas.py}, \texttt{is\_public\_http} in \texttt{backend/Tools/link\_analyzer/route/route.py}.
\item
  Status: Implemented.
\end{itemize}

\hypertarget{path-traversal-protection}{%
\subsubsection{Path Traversal Protection}\label{sec:00-Overview__security-controls-mapping:path-traversal-protection}}

\begin{itemize}
\tightlist
\item
  Control: Block traversal markers and forbidden file extensions.
\item
  Evidence: \texttt{backend/app/\_\_init\_\_.py} request filtering.
\item
  Status: Implemented.
\end{itemize}

\hypertarget{secrets-management}{%
\subsubsection{Secrets Management}\label{sec:00-Overview__security-controls-mapping:secrets-management}}

\begin{itemize}
\tightlist
\item
  Control: Environment-based secrets and API keys.
\item
  Evidence: \texttt{backend/config.py} uses \texttt{decouple} config variables.
\item
  Status: Partially implemented; rotation and vault integration are TBD.
\end{itemize}

\hypertarget{rate-limiting-and-abuse-prevention}{%
\subsubsection{Rate Limiting and Abuse Prevention}\label{sec:00-Overview__security-controls-mapping:rate-limiting-and-abuse-prevention}}

\begin{itemize}
\tightlist
\item
  Control: API rate limiting and quotas.
\item
  Evidence: Not found.
\item
  Status: TBD (verify in code).
\end{itemize}

\begin{longtable}[]{@{}ll@{}}
\toprule
Area & Control \\
\midrule
\endhead
Authentication & JWT access \& refresh tokens \\
Authorization & Role-based access control \\
Input Validation & Server-side validation \\
Sensitive Data & Hash-only processing \\
URL Safety & SSRF target filtering and allow/deny rules \\
File Safety & Static analysis only \\
Auditability & History logging \\
Abuse Prevention & Tool usage counters \\
Privacy & Minimal retention \\
Failure Safety & Risk-increasing failures \\
\bottomrule
\end{longtable}

\FloatBarrier
\hypertarget{authentication-and-authorization}{%
\subsection{Authentication and Authorization}\label{sec:01-Backend__authentication-authorization:authentication-and-authorization}}

\hypertarget{credential-encryption}{%
\subsubsection{Credential Encryption}\label{sec:01-Backend__authentication-authorization:credential-encryption}}

\begin{itemize}
\tightlist
\item
  Frontend encrypts \texttt{username}, \texttt{email}, and \texttt{password} with RSA.
\item
  Public key is fetched from \texttt{/api/auth/public-key}.
\item
  Backend decrypts fields via \texttt{backend/app/security.py} before processing.
\end{itemize}

\hypertarget{signup}{%
\subsubsection{Signup}\label{sec:01-Backend__authentication-authorization:signup}}

\begin{itemize}
\tightlist
\item
  Endpoint: \texttt{POST\ /api/auth/signup}.
\item
  Validates required fields and uniqueness of \texttt{username} and \texttt{email}.
\item
  Stores hashed password (\texttt{werkzeug.security.generate\_password\_hash}).
\end{itemize}

\hypertarget{login}{%
\subsubsection{Login}\label{sec:01-Backend__authentication-authorization:login}}

\begin{itemize}
\tightlist
\item
  Endpoint: \texttt{POST\ /api/auth/login}.
\item
  Enforces account suspension after three failed attempts (\texttt{User.record\_failed\_login}).
\item
  Issues JWT access and refresh tokens with explicit expiry.
\item
  Stores refresh token JTI in the \texttt{RefreshToken} table.
\end{itemize}

\hypertarget{refresh}{%
\subsubsection{Refresh}\label{sec:01-Backend__authentication-authorization:refresh}}

\begin{itemize}
\tightlist
\item
  Endpoint: \texttt{POST\ /api/auth/refresh} (requires refresh token).
\item
  Validates existing refresh token, checks expiry, revokes old token, and issues rotated access + refresh tokens.
\end{itemize}

\hypertarget{logout}{%
\subsubsection{Logout}\label{sec:01-Backend__authentication-authorization:logout}}

\begin{itemize}
\tightlist
\item
  Endpoint: \texttt{POST\ /api/auth/logout} (requires refresh token).
\item
  Revokes the supplied refresh token.
\end{itemize}

\hypertarget{logout-all}{%
\subsubsection{Logout All}\label{sec:01-Backend__authentication-authorization:logout-all}}

\begin{itemize}
\tightlist
\item
  Endpoint: \texttt{POST\ /api/auth/logout-all}.
\item
  Revokes all active refresh tokens for the user.
\end{itemize}

\hypertarget{authorization-model}{%
\subsubsection{Authorization Model}\label{sec:01-Backend__authentication-authorization:authorization-model}}

\begin{itemize}
\tightlist
\item
  Role stored in JWT claims (\texttt{role}).
\item
  \texttt{require\_admin()} enforces admin-only access to protected routes.
\item
  Admin CRUD operations have additional constraints (master admin restrictions in \texttt{backend/app/user.py}).
\end{itemize}

\hypertarget{frontend-session-handling}{%
\subsubsection{Frontend Session Handling}\label{sec:01-Backend__authentication-authorization:frontend-session-handling}}

\begin{itemize}
\tightlist
\item
  Access and refresh tokens are stored in \texttt{localStorage}.
\item
  Client token-refresh middleware renews expired access tokens and updates both tokens on rotation.
\item
  Admin route access is based on stored user role in \texttt{localStorage}.
\end{itemize}

\hypertarget{authentication-and-authorization-1}{%
\subsection{Authentication and Authorization}\label{sec:01-Backend__authentication-authorization:authentication-and-authorization-1}}

\hypertarget{overview}{%
\subsubsection{Overview}\label{sec:01-Backend__authentication-authorization:overview}}

DefendX uses a \textbf{token-based authentication model} built on JSON Web Tokens (JWT), combined with role-based authorization controls.

Authentication and authorization are enforced \textbf{before any security tool logic executes}.

\begin{center}\rule{0.5\linewidth}{0.5pt}\end{center}

\hypertarget{token-types}{%
\subsubsection{Token Types}\label{sec:01-Backend__authentication-authorization:token-types}}

\hypertarget{access-token}{%
\paragraph{Access Token}\label{sec:01-Backend__authentication-authorization:access-token}}

\begin{itemize}
\tightlist
\item
  Short-lived
\item
  Sent in the \texttt{Authorization:\ Bearer\ \textless{}token\textgreater{}} header
\item
  Used for all authenticated API requests
\item
  Not stored server-side
\end{itemize}

\hypertarget{refresh-token}{%
\paragraph{Refresh Token}\label{sec:01-Backend__authentication-authorization:refresh-token}}

\begin{itemize}
\tightlist
\item
  Long-lived
\item
  Stored in the database
\item
  Used to obtain new access tokens
\item
  Explicitly validated and revocable
\end{itemize}

\begin{center}\rule{0.5\linewidth}{0.5pt}\end{center}

\hypertarget{authentication-flow}{%
\subsubsection{Authentication Flow}\label{sec:01-Backend__authentication-authorization:authentication-flow}}

\begin{enumerate}
\def\labelenumi{\arabic{enumi}.}
\tightlist
\item
  User submits credentials
\item
  Backend validates credentials
\item
  Backend issues:

  \begin{itemize}
  \tightlist
  \item
    Access token
  \item
    Refresh token
  \end{itemize}
\item
  Frontend stores and uses access token
\item
  Refresh token used only for token renewal
\end{enumerate}

\begin{center}\rule{0.5\linewidth}{0.5pt}\end{center}

\hypertarget{refresh-token-validation}{%
\subsubsection{Refresh Token Validation}\label{sec:01-Backend__authentication-authorization:refresh-token-validation}}

On refresh requests:

\begin{enumerate}
\def\labelenumi{\arabic{enumi}.}
\tightlist
\item
  Token is decoded
\item
  Token identifier is checked against the database
\item
  Expired or revoked tokens are rejected
\item
  New access token is issued
\end{enumerate}

Refresh tokens are \textbf{validated server-side}, unlike access tokens.

\begin{center}\rule{0.5\linewidth}{0.5pt}\end{center}

\hypertarget{authorization-rbac}{%
\subsubsection{Authorization (RBAC)}\label{sec:01-Backend__authentication-authorization:authorization-rbac}}

Certain endpoints require elevated privileges.

\hypertarget{role-based-access-control}{%
\paragraph{Role-Based Access Control}\label{sec:01-Backend__authentication-authorization:role-based-access-control}}

\begin{itemize}
\tightlist
\item
  User roles are embedded in JWT claims
\item
  Decorators enforce required roles
\item
  Authorization checks occur before route execution
\end{itemize}

Example use cases:

\begin{itemize}
\tightlist
\item
  Administrative actions
\item
  User management
\item
  System-level operations
\end{itemize}

\begin{center}\rule{0.5\linewidth}{0.5pt}\end{center}

\hypertarget{security-properties}{%
\subsubsection{Security Properties}\label{sec:01-Backend__authentication-authorization:security-properties}}

The authentication system ensures:

\begin{itemize}
\tightlist
\item
  Stateless access validation
\item
  Server-controlled session longevity
\item
  Explicit privilege enforcement
\item
  Clear separation of identity sources
\end{itemize}

\begin{center}\rule{0.5\linewidth}{0.5pt}\end{center}

\hypertarget{known-constraints}{%
\subsubsection{Known Constraints}\label{sec:01-Backend__authentication-authorization:known-constraints}}

\begin{itemize}
\tightlist
\item
  Access tokens are not blocklisted
\item
  Refresh tokens are not automatically rotated
\end{itemize}

These constraints are documented and accepted as part of the design.

\FloatBarrier
\hypertarget{security-hardening}{%
\subsection{Security Hardening}\label{sec:01-Backend__security-hardening:security-hardening}}

\hypertarget{request-path-hardening}{%
\subsubsection{Request Path Hardening}\label{sec:01-Backend__security-hardening:request-path-hardening}}

\begin{itemize}
\tightlist
\item
  Traversal markers and forbidden file extensions are blocked in \texttt{backend/app/\_\_init\_\_.py}.
\item
  Static assets are served only from \texttt{frontend/dist} or \texttt{frontend/build} when present.
\end{itemize}

\hypertarget{token-revocation-enforcement}{%
\subsubsection{Token Revocation Enforcement}\label{sec:01-Backend__security-hardening:token-revocation-enforcement}}

\begin{itemize}
\tightlist
\item
  Refresh tokens are stored with JTI and checked via \texttt{@jwt.token\_in\_blocklist\_loader} in \texttt{create\_app()}.
\item
  Access tokens are not currently blocked server-side (only refresh tokens).
\end{itemize}

\hypertarget{url-target-validation}{%
\subsubsection{URL Target Validation}\label{sec:01-Backend__security-hardening:url-target-validation}}

\begin{itemize}
\tightlist
\item
  Link Analyzer validates target URLs, disallowing private/loopback/link-local IPs (\texttt{is\_public\_http}).
\end{itemize}

\hypertarget{credential-handling}{%
\subsubsection{Credential Handling}\label{sec:01-Backend__security-hardening:credential-handling}}

\begin{itemize}
\tightlist
\item
  RSA-based encryption is used for credential fields to reduce exposure in transit within client environments.
\item
  Backend rejects decryption failures for long encrypted-like strings.
\end{itemize}

\hypertarget{cors}{%
\subsubsection{CORS}\label{sec:01-Backend__security-hardening:cors}}

\begin{itemize}
\tightlist
\item
  Public key endpoint has explicit origin allowlist for local dev.
\item
  Global CORS allows all origins (\texttt{*}) for other routes (verify if this is desired for production).
\end{itemize}

\hypertarget{file-handling}{%
\subsubsection{File Handling}\label{sec:01-Backend__security-hardening:file-handling}}

\begin{itemize}
\tightlist
\item
  File upload analysis writes to OS temp directory and removes files after analysis.
\end{itemize}

\hypertarget{security-gaps-tbd}{%
\subsubsection{Security Gaps (TBD)}\label{sec:01-Backend__security-hardening:security-gaps-tbd}}

\begin{itemize}
\tightlist
\item
  Rate limiting and abuse prevention are not evident in code (TBD (verify in code)).
\item
  CSRF protection and secure cookie strategy are not used because tokens are stored in \texttt{localStorage} (risk: XSS). TBD (verify in code).
\item
  Production TLS termination and WAF configuration are not defined in code (TBD (verify in code)).
\end{itemize}

\hypertarget{backend-security-hardening}{%
\subsection{Backend Security Hardening}\label{sec:01-Backend__security-hardening:backend-security-hardening}}

\hypertarget{purpose}{%
\subsubsection{Purpose}\label{sec:01-Backend__security-hardening:purpose}}

This document describes \textbf{request-level and application-level security controls} enforced by the DefendX backend to reduce attack surface and prevent common exploitation techniques.

\begin{center}\rule{0.5\linewidth}{0.5pt}\end{center}

\hypertarget{request-filtering}{%
\subsubsection{Request Filtering}\label{sec:01-Backend__security-hardening:request-filtering}}

A global request filter is applied before route handling.

Blocked patterns include:

\begin{itemize}
\tightlist
\item
  Path traversal attempts (\texttt{../}, encoded variants)
\item
  Requests targeting sensitive directories:

  \begin{itemize}
  \tightlist
  \item
    \texttt{.git}
  \item
    \texttt{node\_modules}
  \end{itemize}
\item
  Requests for source code files:

  \begin{itemize}
  \tightlist
  \item
    \texttt{.py}
  \item
    \texttt{.ts}
  \item
    \texttt{.env}
  \item
    \texttt{.json} (non-API)
  \end{itemize}
\end{itemize}

Requests matching these patterns are rejected immediately.

\begin{center}\rule{0.5\linewidth}{0.5pt}\end{center}

\hypertarget{input-validation}{%
\subsubsection{Input Validation}\label{sec:01-Backend__security-hardening:input-validation}}

All inputs are validated server-side, including:

\begin{itemize}
\tightlist
\item
  URLs
\item
  File uploads
\item
  Email identifiers
\item
  Credential inputs
\end{itemize}

Client-side validation is advisory only.

\begin{center}\rule{0.5\linewidth}{0.5pt}\end{center}

\hypertarget{cors-policy}{%
\subsubsection{CORS Policy}\label{sec:01-Backend__security-hardening:cors-policy}}

CORS is enabled to support frontend integration.

Characteristics:

\begin{itemize}
\tightlist
\item
  Controlled origins
\item
  Explicit method allowances
\item
  Header restrictions
\end{itemize}

Sensitive endpoints may have stricter rules.

\begin{center}\rule{0.5\linewidth}{0.5pt}\end{center}

\hypertarget{sensitive-data-handling}{%
\subsubsection{Sensitive Data Handling}\label{sec:01-Backend__security-hardening:sensitive-data-handling}}

The backend enforces:

\begin{itemize}
\tightlist
\item
  No plaintext credential storage
\item
  No raw email persistence
\item
  No file execution
\item
  Minimal metadata retention
\end{itemize}

Secrets are never logged or returned in responses.

\begin{center}\rule{0.5\linewidth}{0.5pt}\end{center}

\hypertarget{external-api-safety}{%
\subsubsection{External API Safety}\label{sec:01-Backend__security-hardening:external-api-safety}}

When interacting with external services:

\begin{itemize}
\tightlist
\item
  Timeouts are enforced
\item
  Failures are handled explicitly
\item
  Missing data increases risk instead of returning ``safe''
\end{itemize}

External service availability is treated as untrusted.

\begin{center}\rule{0.5\linewidth}{0.5pt}\end{center}

\hypertarget{ssrf-considerations}{%
\subsubsection{SSRF Considerations}\label{sec:01-Backend__security-hardening:ssrf-considerations}}

URL-based tools enforce:

\begin{itemize}
\tightlist
\item
  Scheme restrictions (HTTP/HTTPS)
\item
  Hostname-based public target validation
\end{itemize}

Limitations:

\begin{itemize}
\tightlist
\item
  Hostname resolution is not validated against private IP ranges
\end{itemize}

Mitigation relies on:

\begin{itemize}
\tightlist
\item
  Conservative scoring
\item
  Deployment-level egress controls
\end{itemize}

\begin{center}\rule{0.5\linewidth}{0.5pt}\end{center}

\hypertarget{failure-containment}{%
\subsubsection{Failure Containment}\label{sec:01-Backend__security-hardening:failure-containment}}

Failures in non-critical subsystems (e.g., logging) do not:

\begin{itemize}
\tightlist
\item
  Block security analysis
\item
  Produce false success states
\end{itemize}

The system prefers partial visibility over total failure.

\begin{center}\rule{0.5\linewidth}{0.5pt}\end{center}

\hypertarget{security-posture-summary}{%
\subsubsection{Security Posture Summary}\label{sec:01-Backend__security-hardening:security-posture-summary}}

The backend applies:

\begin{itemize}
\tightlist
\item
  Defense in depth
\item
  Least privilege
\item
  Explicit trust boundaries
\item
  Conservative failure handling
\end{itemize}

\FloatBarrier
\hypertarget{frontend-security-model}{%
\subsection{Frontend Security Model}\label{sec:04-Frontend__frontend-security-model:frontend-security-model}}

\hypertarget{token-storage}{%
\subsubsection{Token Storage}\label{sec:04-Frontend__frontend-security-model:token-storage}}

\begin{itemize}
\tightlist
\item
  Access and refresh tokens are stored in \texttt{localStorage}.
\item
  Risk: XSS can expose tokens. No in-code mitigation beyond standard frontend practices (Not evidenced in the reviewed codebase).
\end{itemize}

\hypertarget{credential-encryption}{%
\subsubsection{Credential Encryption}\label{sec:04-Frontend__frontend-security-model:credential-encryption}}

\begin{itemize}
\tightlist
\item
  RSA encryption is applied to \texttt{username}, \texttt{email}, and \texttt{password} fields before submit.
\item
  Public key is fetched from \texttt{/api/auth/public-key}.
\end{itemize}

\hypertarget{session-management}{%
\subsubsection{Session Management}\label{sec:04-Frontend__frontend-security-model:session-management}}

\begin{itemize}
\tightlist
\item
  Access token is attached to all API requests.
\item
  Refresh token is used only for \texttt{/auth/refresh} and replaced on rotation.
\end{itemize}

\hypertarget{route-protection}{%
\subsubsection{Route Protection}\label{sec:04-Frontend__frontend-security-model:route-protection}}

\begin{itemize}
\tightlist
\item
  Protected routes require presence of access token.
\item
  Admin routes require role check from user payload.
\end{itemize}

\hypertarget{error-handling}{%
\subsubsection{Error Handling}\label{sec:04-Frontend__frontend-security-model:error-handling}}

\begin{itemize}
\tightlist
\item
  API errors are parsed and surfaced to UI with descriptive messages.
\end{itemize}

\hypertarget{cors-and-api-base-url}{%
\subsubsection{CORS and API Base URL}\label{sec:04-Frontend__frontend-security-model:cors-and-api-base-url}}

\begin{itemize}
\tightlist
\item
  Default API base is \texttt{/api}, aligning with backend proxy or same-origin deployment.
\end{itemize}

\hypertarget{trust-assumptions}{%
\subsubsection{Trust Assumptions}\label{sec:04-Frontend__frontend-security-model:trust-assumptions}}

The frontend is treated as \textbf{fully untrusted}.

Assumptions:

\begin{itemize}
\tightlist
\item
  Client-side logic can be modified
\item
  Client-side validation can be bypassed
\item
  Requests may be forged or replayed
\end{itemize}

All security enforcement must occur server-side.

\begin{center}\rule{0.5\linewidth}{0.5pt}\end{center}

\hypertarget{token-handling}{%
\subsubsection{Token Handling}\label{sec:04-Frontend__frontend-security-model:token-handling}}

\hypertarget{access-tokens}{%
\paragraph{Access Tokens}\label{sec:04-Frontend__frontend-security-model:access-tokens}}

\begin{itemize}
\tightlist
\item
  Short-lived JWTs
\item
  Sent in Authorization headers
\item
  Never embedded in URLs
\end{itemize}

\hypertarget{refresh-tokens}{%
\paragraph{Refresh Tokens}\label{sec:04-Frontend__frontend-security-model:refresh-tokens}}

\begin{itemize}
\tightlist
\item
  Not directly accessed by frontend JavaScript when possible
\item
  Used only for token renewal
\item
  Validated server-side
\end{itemize}

Exact storage strategy may vary by frontend implementation.

\begin{center}\rule{0.5\linewidth}{0.5pt}\end{center}

\hypertarget{recommended-token-storage-strategies}{%
\subsubsection{Recommended Token Storage Strategies}\label{sec:04-Frontend__frontend-security-model:recommended-token-storage-strategies}}

\hypertarget{in-memory-storage}{%
\paragraph{In-Memory Storage}\label{sec:04-Frontend__frontend-security-model:in-memory-storage}}

\begin{itemize}
\tightlist
\item
  Reduced XSS exposure
\item
  Token lost on refresh
\end{itemize}

\hypertarget{secure-cookies-http-only}{%
\paragraph{Secure Cookies (HTTP-only)}\label{sec:04-Frontend__frontend-security-model:secure-cookies-http-only}}

\begin{itemize}
\tightlist
\item
  Strong XSS protection
\item
  Requires CSRF protections
\end{itemize}

LocalStorage is discouraged for sensitive tokens.

\begin{center}\rule{0.5\linewidth}{0.5pt}\end{center}

\hypertarget{client-side-validation}{%
\subsubsection{Client-Side Validation}\label{sec:04-Frontend__frontend-security-model:client-side-validation}}

Client-side validation is used for:

\begin{itemize}
\tightlist
\item
  UX improvement
\item
  Early feedback
\item
  Preventing obvious mistakes
\end{itemize}

Client-side validation is \textbf{never authoritative}.

\begin{center}\rule{0.5\linewidth}{0.5pt}\end{center}

\hypertarget{cors-and-browser-security}{%
\subsubsection{CORS and Browser Security}\label{sec:04-Frontend__frontend-security-model:cors-and-browser-security}}

The frontend relies on backend CORS configuration.

Security considerations:

\begin{itemize}
\tightlist
\item
  Strict origin configuration
\item
  Limited allowed methods
\item
  Controlled headers
\end{itemize}

Browser-enforced policies complement backend controls.

\begin{center}\rule{0.5\linewidth}{0.5pt}\end{center}

\hypertarget{error-and-failure-safety}{%
\subsubsection{Error and Failure Safety}\label{sec:04-Frontend__frontend-security-model:error-and-failure-safety}}

Frontend behavior under failure:

\begin{itemize}
\tightlist
\item
  Does not assume safety when errors occur
\item
  Displays partial results explicitly
\item
  Warns users when intelligence data is unavailable
\end{itemize}

\begin{center}\rule{0.5\linewidth}{0.5pt}\end{center}

\hypertarget{prevented-attack-classes}{%
\subsubsection{Prevented Attack Classes}\label{sec:04-Frontend__frontend-security-model:prevented-attack-classes}}

By design, the frontend does not allow:

\begin{itemize}
\tightlist
\item
  Privilege escalation via UI
\item
  Direct external intelligence API access
\item
  Trust decisions based on client state
\end{itemize}

\begin{center}\rule{0.5\linewidth}{0.5pt}\end{center}

\hypertarget{security-responsibilities-summary}{%
\subsubsection{Security Responsibilities Summary}\label{sec:04-Frontend__frontend-security-model:security-responsibilities-summary}}

\begin{longtable}[]{@{}ll@{}}
\toprule
Responsibility & Location \\
\midrule
\endhead
Authentication & Backend \\
Authorization & Backend \\
Risk Scoring & Backend \\
Input Validation & Backend \\
Secret Storage & Backend \\
Visualization & Frontend \\
\bottomrule
\end{longtable}

\begin{center}\rule{0.5\linewidth}{0.5pt}\end{center}

\hypertarget{security-model-summary}{%
\subsubsection{Security Model Summary}\label{sec:04-Frontend__frontend-security-model:security-model-summary}}

The frontend enhances usability while preserving strict trust boundaries.

Compromise of the frontend does not imply compromise of DefendX's security logic.

\FloatBarrier

\chapter{Link Analyzer}
\section{Overview}
The Link Analyzer evaluates a submitted URL and returns a risk-oriented result. The tool combines URL checks, reputation lookups, and optional deeper inspection paths to estimate whether a link is likely safe or suspicious.

This tool matters for social engineering defense because malicious campaigns often start with a URL. Early link triage helps users and analysts block phishing pages, malware delivery links, and impersonation attempts before account or device compromise.

\begin{table}[H]
  \centering
  \caption{Link Analyzer API Endpoints}
  \begin{tabular}{|l|p{0.62\textwidth}|}
    \hline
    \textbf{Endpoint} & \textbf{Description} \\ \hline
    \texttt{POST /quick-scan} & Fast URL inspection \\ \hline
    \texttt{POST /deep-scan} & Deeper URL inspection with enrichment \\ \hline
    \texttt{POST /custom-scan} & Custom scan options \\ \hline
  \end{tabular}
\end{table}

\begin{figure}[H]
  \centering
  \includegraphics[width=\linewidth]{figures/fig-authentication-jwt-sequence.png}
  \caption{Authentication workflow and JWT-protected request sequence for Link Analyzer execution.}
  \label{fig:link-analyzer-request-response-flow}
\end{figure}
\FloatBarrier

\section{Scope and Assumptions}
\begin{itemize}
\item Supported input is a URL in request payload. Scan modes include quick, deep, and custom paths.
\item Out of scope: full browser sandboxing guarantees, user education workflow, and policy enforcement outside this API tool.
\item Trust boundaries: user input is untrusted; external intelligence services are semi-trusted; internal scoring logic is trusted system code.
\end{itemize}

\section{Inputs and Outputs}
\textbf{Inputs}
\begin{itemize}
\item URL field: required, non-empty.
\item Scheme expected: HTTP/HTTPS.
\item URL normalization details: Not evidenced in the reviewed codebase.
\item Size and payload limits for request body: Not evidenced in the reviewed codebase.
\item Custom scan option accepts selected checks list.
\end{itemize}

\textbf{Outputs}
\begin{itemize}
\item Risk object with score, level, and reasons.
\item Check-level details (domain/reputation/redirect/SSL and optional deep checks).
\item Indicator flags and metadata fields: exact schema Not evidenced in the reviewed codebase.
\end{itemize}

\begin{figure}[H]
  \centering
  \includegraphics[width=\linewidth]{figures/ui-link-analyzer-input.png}
  \caption{Link Analyzer user interface (URL input and scan modes). UI label reflects an earlier prototype name; in this report it is referred to as Link Analyzer.}
  \label{fig:link-analyzer-ui-input}
\end{figure}
\FloatBarrier

\begin{figure}[H]
  \centering
  \includegraphics[width=\linewidth]{figures/ui-link-analyzer-custom-scan-options.png}
  \caption{Link Analyzer custom scan configuration interface with selectable inspection options.}
  \label{fig:link-analyzer-ui-custom-scan}
\end{figure}
\FloatBarrier

\begin{figure}[H]
  \centering
  \includegraphics[width=\linewidth]{figures/ui-link-analyzer-result-low.png}
  \caption{Link Analyzer result interface showing a low-risk classification.}
  \label{fig:link-analyzer-ui-low}
\end{figure}
\FloatBarrier

\begin{figure}[H]
  \centering
  \includegraphics[width=\linewidth]{figures/ui-link-analyzer-result-review.png}
  \caption{Link Analyzer result interface showing a needs-review classification.}
  \label{fig:link-analyzer-ui-review}
\end{figure}
\FloatBarrier

\begin{figure}[H]
  \centering
  \includegraphics[width=\linewidth]{figures/ui-link-analyzer-result-high.png}
  \caption{Link Analyzer result interface showing a high-risk classification.}
  \label{fig:link-analyzer-ui-high}
\end{figure}
\FloatBarrier

\begin{figure}[H]
  \centering
  \includegraphics[width=\linewidth]{figures/fig-backend-processing-pipeline.png}
  \caption{Link Analyzer: Redirect and Reputation Pipeline}
  \label{fig:link-analyzer-redirect-reputation-pipeline}
\end{figure}
\FloatBarrier

\begin{figure}[H]
  \centering
  \includegraphics[width=\linewidth]{figures/fig-system-architecture-overview.png}
  \caption{Link Analyzer: External intelligence inputs and resulting tool output indicators in the system architecture context.}
  \label{fig:link-analyzer-threat-indicators-output}
\end{figure}
\FloatBarrier

\section{Processing Pipeline}
\begin{enumerate}
\item Accept and validate URL request.
\item Apply target safety checks before outbound processing.
\item Execute selected checks (quick, deep, or custom path).
\item Collect check outputs and compute aggregate risk.
\item Return structured response and record tool usage.
\end{enumerate}

\textbf{Error handling behavior}
\begin{itemize}
\item Invalid input is rejected before analysis.
\item External API timeout or failure handling policy: Not evidenced in the reviewed codebase.
\item Missing data from one check is handled conservatively in scoring.
\end{itemize}

\textbf{Logging and audit}
\begin{itemize}
\item History entries and tool counters are updated per scan.
\item Sensitive payload retention policy details: Not evidenced in the reviewed codebase.
\item Authentication secrets are not expected in this tool input.
\end{itemize}

\section{Detection Logic}
\begin{itemize}
\item Indicators include domain age, redirect behavior, SSL status, IP/geolocation context, and reputation signals.
\item Optional deep checks include dynamic behavior, favicon signals, and brand mismatch indicators.
\item Score combination logic is rule-based/weighted in practice, but exact weights and thresholds are Not evidenced in the reviewed codebase.
\end{itemize}

\section{Security and Abuse Resistance}
\begin{itemize}
\item Input is validated and normalized before processing.
\item URL target filtering is used to reduce SSRF exposure, including private/loopback target rejection.
\item Rate limiting and abuse controls at gateway/API level: Not evidenced in the reviewed codebase.
\item Tool usage counters support abuse monitoring.
\end{itemize}

\section{Privacy Considerations}
\begin{itemize}
\item Tool accepts URL input and returns analysis output.
\item Stored versus transient fields in history are deployment-dependent: Not evidenced in the reviewed codebase.
\item Token handling relies on platform authentication layer, not this tool-specific logic.
\end{itemize}

\section{Limitations}
\begin{itemize}
\item Results depend on external service availability and quality.
\item Dynamic checks may not cover all evasive behavior.
\item SSRF protections are hostname/target-rule based and require strict maintenance.
\item Risk score is an analyst aid, not proof of compromise.
\end{itemize}

\section{Test Plan (Scalable)}
\textbf{Unit tests}
\begin{itemize}
\item URL parser and validator edge cases.
\item Score calculation for isolated indicator combinations.
\item Check selector behavior for custom scan mode.
\end{itemize}

\textbf{Integration tests}
\begin{itemize}
\item End-to-end API response structure for each scan mode.
\item External API success/failure fallback behavior.
\end{itemize}

\textbf{Adversarial tests}
\begin{itemize}
\item Redirect chains, homograph-like domains, and suspicious TLS setups.
\item Known phishing URLs and benign controls.
\end{itemize}

\textbf{Performance tests}
\begin{itemize}
\item Latency by scan mode.
\item Timeout behavior under upstream API delays.
\end{itemize}

\section{Operational Notes}
\begin{itemize}
\item Monitor scan failures, timeout rates, and risk-level distributions.
\item Monitor tool counter growth for usage and abuse patterns.
\item Troubleshooting steps: verify API keys, check upstream service status, inspect validation failures, and review history entries.
\end{itemize}

\subsection{Audit Trail and History Logging}
\begin{figure}[H]
  \centering
  \includegraphics[width=\linewidth]{figures/history-link-analyzer-filter.png}
  \caption{Audit trail showing historical Link Analyzer executions filtered by tool.}
  \label{fig:history-link-analyzer}
\end{figure}
\FloatBarrier

\FloatBarrier

\chapter{File Download Checker}
\section{Overview}
The File Download Checker evaluates uploaded files and estimates whether they are unsafe. The tool performs static inspection and reputation lookup, then returns a structured risk result.

This tool matters for social engineering defense because file-based lures are common in phishing and fraud campaigns. Static triage helps reduce exposure to malicious attachments before users open them.

Some UI screenshots in this chapter reflect earlier prototype naming; in this report the tool is referred to as File Download Checker.

\begin{figure}[htbp]
  \centering
  \includegraphics[width=\linewidth]{figures/fig-file-checker-processing-flow}
  \caption{File Download Checker processing flow}
  \label{fig:file-checker-processing-flow}
\end{figure}
\FloatBarrier

Figure~\ref{fig:file-checker-processing-flow} summarizes the processing sequence used to transform an uploaded file into a risk result.

\begin{table}[H]
  \centering
  \caption{File Download Checker API Endpoints}
  \begin{tabular}{|l|p{0.62\textwidth}|}
    \hline
    \textbf{Endpoint} & \textbf{Description} \\ \hline
    \texttt{POST /tools/file-checker} & Upload file for static analysis \\ \hline
  \end{tabular}
\end{table}

Figure~\ref{fig:file-checker-endpoint} shows the endpoint representation used in implementation documentation.

\begin{figure}[htbp]
  \centering
  \includegraphics[width=\linewidth]{figures/fig-file-checker-endpoint}
  \caption{File Download Checker API endpoint}
  \label{fig:file-checker-endpoint}
\end{figure}

\begin{figure}[htbp]
  \centering
  \includegraphics[width=\linewidth]{figures/fig-file-checker-static-analysis-pipeline}
  \caption{File Download Checker static analysis pipeline}
  \label{fig:file-checker-static-analysis-pipeline}
\end{figure}
\FloatBarrier

Figure~\ref{fig:file-checker-static-analysis-pipeline} details the static pipeline from initial file parsing to risk feature extraction.

\section{Scope and Assumptions}
\begin{itemize}
\item Supported input is uploaded file data through multipart form requests.
\item Out of scope: dynamic sandbox execution and runtime behavior detonation.
\item Trust boundaries: uploaded file is untrusted; malware intelligence service is semi-trusted; scoring and validation code are trusted system components.
\end{itemize}

\section{Inputs and Outputs}
\textbf{Inputs}
\begin{itemize}
\item Multipart file field is required.
\item File size limits are enforced, exact thresholds Not evidenced in the reviewed codebase.
\item Allowed/restricted file types list: Not evidenced in the reviewed codebase.
\item Filename normalization and sanitization are applied.
\end{itemize}

\textbf{Outputs}
\begin{itemize}
\item File metadata (name, extension, MIME type).
\item Hash and static indicators (including entropy and type-specific signals where available).
\item Risk result with score, level, and reasons.
\item Exact response schema and optional fields: Not evidenced in the reviewed codebase.
\end{itemize}

\begin{figure}[H]
  \centering
  \includegraphics[width=\linewidth]{figures/ui-file-checker-upload.png}
  \caption{File Download Checker user interface (upload / drop zone).}
  \label{fig:file-checker-ui-upload}
\end{figure}
\FloatBarrier

\begin{figure}[H]
  \centering
  \includegraphics[width=\linewidth]{figures/ui-file-checker-result-low.png}
  \caption{File Download Checker result interface showing a low-risk classification.}
  \label{fig:file-checker-ui-low}
\end{figure}
\FloatBarrier

\begin{figure}[H]
  \centering
  \includegraphics[width=\linewidth]{figures/ui-file-checker-result-suspicious.png}
  \caption{File Download Checker result interface showing a suspicious classification with triggered indicators.}
  \label{fig:file-checker-ui-suspicious}
\end{figure}
\FloatBarrier

\begin{figure}[H]
  \centering
  \includegraphics[width=\linewidth]{figures/ui-file-checker-result-suspicious.png}
  \caption{File Download Checker: Example Result Screen}
  \label{fig:file-checker-example-result-screen}
\end{figure}
\FloatBarrier

\section{Processing Pipeline}
\begin{enumerate}
\item Validate upload request and file presence.
\item Sanitize metadata and apply size/type pre-checks.
\item Store file temporarily for analysis.
\item Run static analysis and compute hash.
\item Query reputation service using computed hash.
\item Aggregate indicators into risk output and return response.
\item Record usage and cleanup temporary artifacts.
\end{enumerate}

\textbf{Error handling behavior}
\begin{itemize}
\item Invalid or missing file input is rejected immediately.
\item Unsupported file states return controlled error output.
\item External reputation API failure behavior: Not evidenced in the reviewed codebase.
\item Temporary file cleanup failure handling: Not evidenced in the reviewed codebase.
\end{itemize}

\textbf{Logging and audit}
\begin{itemize}
\item History and tool counter are updated per request.
\item Full file content should not be logged.
\item Logged metadata fields are deployment-dependent: Not evidenced in the reviewed codebase.
\end{itemize}

\section{Detection Logic}
\begin{itemize}
\item Indicators include MIME/extension consistency, entropy signals, static suspicious patterns, and reputation verdicts.
\item Hash-based lookup contributes known-malicious or unknown status context.
\item Scoring appears rule-based with thresholds; exact weighting and cutoffs are Not evidenced in the reviewed codebase.
\end{itemize}

\section{Security and Abuse Resistance}
\begin{itemize}
\item Input hardening checks file presence, size, and sanitized names.
\item Files are analyzed statically and are not executed.
\item Abuse controls such as rate limits and per-user quotas: Not evidenced in the reviewed codebase.
\item Tool counters support anomaly tracking.
\end{itemize}

\section{Privacy Considerations}
\begin{itemize}
\item Uploaded files are processed for analysis and may use temporary storage.
\item Retention and deletion timing depend on deployment configuration.
\item Authentication tokens are handled at platform level, not by file content logic.
\end{itemize}

\section{Limitations}
\begin{itemize}
\item Static-only analysis can miss behavior that appears only at runtime.
\item Reputation coverage is limited for new or rare files.
\item Encrypted or obfuscated payloads can reduce indicator quality.
\item Final score supports triage; it is not a full malware verdict.
\end{itemize}

\section{Test Plan (Scalable)}
\textbf{Unit tests}
\begin{itemize}
\item Validator behavior for file size/type/empty input.
\item Hashing and metadata extraction correctness.
\item Score mapping for known indicator combinations.
\end{itemize}

\textbf{Integration tests}
\begin{itemize}
\item End-to-end upload and response schema verification.
\item External reputation lookup success/failure paths.
\end{itemize}

\textbf{Adversarial tests}
\begin{itemize}
\item Renamed extensions and MIME mismatch samples.
\item Packed/obfuscated files and malformed payload cases.
\end{itemize}

\textbf{Performance tests}
\begin{itemize}
\item Latency versus file size and type.
\item Throughput and timeout behavior at concurrent load.
\end{itemize}

\section{Operational Notes}
\begin{itemize}
\item Monitor upload failures, analysis errors, and reputation lookup availability.
\item Track tool usage counters and distribution of risk levels.
\item Troubleshooting steps: check file limits, inspect temp storage health, confirm API key validity, and review history entries.
\end{itemize}

\subsection{Audit Trail and History Logging}
\begin{figure}[H]
  \centering
  \includegraphics[width=\linewidth]{figures/history-file-checker-filter.png}
  \caption{Audit trail showing historical File Download Checker scans filtered by tool.}
  \label{fig:history-file-checker}
\end{figure}
\FloatBarrier

\FloatBarrier

\chapter{Evaluation}
This chapter presents a structured discussion of observed system behavior under representative usage conditions. The evaluation is focused on functional outcomes, consistency of risk signaling, and practical constraints of the implemented workflow.\par

\section{Test Scenarios}
Evaluation scenarios were defined to reflect realistic user actions for both implemented tools. For the Link Analyzer, scenarios include benign URLs, suspicious links requiring manual review, and clearly high-risk links. For the File Download Checker, scenarios include files with low-risk characteristics and files that trigger suspicious static indicators. In each case, the objective is to verify whether the returned risk level and explanatory output remain coherent with the observed input characteristics.\par

\section{Link Analyzer Results}
The Link Analyzer produces tiered results that differentiate low-risk, review-required, and high-risk link submissions. Observed outputs indicate that the tool provides a consistent risk classification format, accompanied by supporting indicators that improve interpretability for end users and analysts. The result presentation supports triage by combining a categorical risk level with concise rationale rather than relying on a single opaque score.\par

\section{File Download Checker Results}
The File Download Checker returns structured output for uploaded files, including metadata, analysis indicators, and a summarized risk decision. Across the tested scenarios, the tool distinguishes low-risk files from suspicious submissions using static evidence available at scan time. The response format remains suitable for operational review because it links the final classification to specific observed attributes.\par

\section{System Limitations}
The current evaluation remains limited in scope and does not yet provide large-scale statistical validation. Results are influenced by static analysis boundaries, the availability and quality of external intelligence, and environment-specific deployment settings. Accordingly, the presented findings should be interpreted as implementation-level validation of workflow behavior rather than exhaustive performance benchmarking.\par

\chapter{Conclusion and Future Work}
\section*{Conclusion}
DefendX provides a practical architecture for detecting social engineering risk in user-facing workflows. The current implementation demonstrates clear request flow, security boundaries, and explainable outputs for two production-relevant tools: Link Analyzer and File Download Checker.\par

\section*{Future Work}
\addcontentsline{toc}{section}{Future Work}
\hypertarget{future-work}{%
\subsection{Future Work}\label{sec:00-Overview__future-work:future-work}}

\begin{itemize}
\tightlist
\item
  Implement backend APIs for tools present in the frontend only (USB, PCAP, Audio, Behavioral Risk). TBD (verify in code).
\item
  Add API rate limiting, request quotas, and abuse detection. TBD (verify in code).
\item
  Provide production configuration (\texttt{prodconfig}) with secure database, secret management, and TLS guidance. TBD (verify in code).
\item
  Harden token storage and client-side security (for example XSS protection strategy). TBD (verify in code).
\item
  Expand audit logs with request metadata, tool inputs, and admin actions where appropriate. TBD (verify in code).
\item
  Formalize Playwright sandboxing and resource limits for dynamic scans. TBD (verify in code).
\end{itemize}

\hypertarget{stronger-ssrf-protections}{%
\subsubsection{Stronger SSRF Protections}\label{sec:00-Overview__future-work:stronger-ssrf-protections}}

Resolve hostnames and block private IP ranges.

\hypertarget{dynamic-file-analysis}{%
\subsubsection{Dynamic File Analysis}\label{sec:00-Overview__future-work:dynamic-file-analysis}}

Optional sandbox-based execution.

\hypertarget{enhanced-url-scoring}{%
\subsubsection{Enhanced URL Signal Scoring}\label{sec:00-Overview__future-work:enhanced-url-scoring}}

Expanded URL signal normalization and weighting.

\hypertarget{expanded-reputation-coverage}{%
\subsubsection{Expanded Reputation Coverage}\label{sec:00-Overview__future-work:expanded-reputation-coverage}}

Additional defensive reputation data sources for URL and file analysis.

\hypertarget{configurable-risk-scoring}{%
\subsubsection{Configurable Risk Scoring}\label{sec:00-Overview__future-work:configurable-risk-scoring}}

Externalized scoring weights.

\hypertarget{native-metrics}{%
\subsubsection{Native Metrics}\label{sec:00-Overview__future-work:native-metrics}}

Tool latency and error metrics.

\hypertarget{api-versioning}{%
\subsubsection{API Versioning}\label{sec:00-Overview__future-work:api-versioning}}

Introduce versioned endpoints.

\FloatBarrier
\hypertarget{known-limitations}{%
\subsection{Known Limitations}\label{sec:00-Overview__known-limitations:known-limitations}}

\begin{itemize}
\tightlist
\item
  Tool coverage mismatch: Frontend routes exist for USB Analyzer, PCAP Analyzer, Audio Analyzer, and Behavioral Risk Analyzer, but corresponding backend APIs were not found in this repository. Marked as TBD (verify in code).
\item
  No explicit rate limiting: No rate limit or throttling middleware is present in the backend code (TBD (verify in code)).
\item
  Production configuration: \texttt{prodconfig} is empty in \texttt{backend/config.py}. Production DB settings and secrets are TBD (verify in code).
\item
  Playwright dependency: Link Analyzer deep/custom scans rely on Playwright. If Playwright is not installed or reachable, dynamic analysis may fail or be skipped (TBD (verify in code)).
\item
  SQLite dev/test: Default configs are SQLite for development and testing, which is not suitable for production workload.
\item
  Frontend token storage: Tokens are stored in \texttt{localStorage}, which is vulnerable to XSS (mitigations not shown in code).
\end{itemize}

\hypertarget{code-alignment-notes}{%
\subsubsection{Code Alignment Notes}\label{sec:00-Overview__known-limitations:code-alignment-notes}}

\begin{itemize}
\tightlist
\item
  \texttt{frontend/src/api/axios.js} defaults to \texttt{/api} base URL, while root README mentions \texttt{VITE\_API\_URL} defaulting to \texttt{http://localhost:5000/api}. Actual default is relative \texttt{/api} (TBD: confirm intended behavior).
\item
  \texttt{frontend/src/api/index.js} references \texttt{/tools/user-behavior/analyze}, but no backend endpoint was found (TBD (verify in code)).
\end{itemize}

\hypertarget{static-file-analysis-only}{%
\subsubsection{Static File Analysis Only}\label{sec:00-Overview__known-limitations:static-file-analysis-only}}

Files are not dynamically executed or detonated.

\hypertarget{hostname-based-ssrf-protection}{%
\subsubsection{Hostname-Based SSRF Protection}\label{sec:00-Overview__known-limitations:hostname-based-ssrf-protection}}

Public target validation does not resolve IP addresses.

\hypertarget{ssrf-edge-case-conservatism}{%
\subsubsection{SSRF Edge-Case Conservatism}\label{sec:00-Overview__known-limitations:ssrf-edge-case-conservatism}}

Hostname-based SSRF protection without full DNS resolution may be conservative in some edge cases.

\hypertarget{deployment-dependent-file-retention}{%
\subsubsection{Deployment-Dependent File Retention}\label{sec:00-Overview__known-limitations:deployment-dependent-file-retention}}

Uploaded files may persist temporarily.

\hypertarget{external-dependency-reliance}{%
\subsubsection{External Dependency Reliance}\label{sec:00-Overview__known-limitations:external-dependency-reliance}}

Reputation APIs may be unavailable or rate-limited.

\hypertarget{heuristic-risk-scoring}{%
\subsubsection{Heuristic Risk Scoring}\label{sec:00-Overview__known-limitations:heuristic-risk-scoring}}

Risk scores are rule-based, not machine-learning-driven.

\FloatBarrier

\appendix
\chapter{Other Tools (Out of Scope)}
Other DefendX tools may exist in the broader platform roadmap. They are intentionally excluded from the main thesis chapters to keep this report focused on Link Analyzer and File Download Checker. Future revisions can add those tools as independent chapters once implementation and evaluation are complete.\par

\cleardoublepage
\addcontentsline{toc}{chapter}{References}
\nocite{*}
\bibliographystyle{IEEEtran}
\bibliography{references}

\end{document}
