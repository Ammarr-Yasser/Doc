\hypertarget{frontend-security-model}{%
\subsection{Frontend Security Model}\label{sec:04-Frontend__frontend-security-model:frontend-security-model}}

\hypertarget{token-storage}{%
\subsubsection{Token Storage}\label{sec:04-Frontend__frontend-security-model:token-storage}}

\begin{itemize}
\tightlist
\item
  Access and refresh tokens are stored in \texttt{localStorage}.
\item
  Risk: XSS can expose tokens. No in-code mitigation beyond standard frontend practices (Not evidenced in the reviewed codebase).
\end{itemize}

\hypertarget{credential-encryption}{%
\subsubsection{Credential Encryption}\label{sec:04-Frontend__frontend-security-model:credential-encryption}}

\begin{itemize}
\tightlist
\item
  RSA encryption is applied to \texttt{username}, \texttt{email}, and \texttt{password} fields before submit.
\item
  Public key is fetched from \texttt{/api/auth/public-key}.
\end{itemize}

\hypertarget{session-management}{%
\subsubsection{Session Management}\label{sec:04-Frontend__frontend-security-model:session-management}}

\begin{itemize}
\tightlist
\item
  Access token is attached to all API requests.
\item
  Refresh token is used only for \texttt{/auth/refresh} and replaced on rotation.
\end{itemize}

\hypertarget{route-protection}{%
\subsubsection{Route Protection}\label{sec:04-Frontend__frontend-security-model:route-protection}}

\begin{itemize}
\tightlist
\item
  Protected routes require presence of access token.
\item
  Admin routes require role check from user payload.
\end{itemize}

\hypertarget{error-handling}{%
\subsubsection{Error Handling}\label{sec:04-Frontend__frontend-security-model:error-handling}}

\begin{itemize}
\tightlist
\item
  API errors are parsed and surfaced to UI with descriptive messages.
\end{itemize}

\hypertarget{cors-and-api-base-url}{%
\subsubsection{CORS and API Base URL}\label{sec:04-Frontend__frontend-security-model:cors-and-api-base-url}}

\begin{itemize}
\tightlist
\item
  Default API base is \texttt{/api}, aligning with backend proxy or same-origin deployment.
\end{itemize}

\hypertarget{trust-assumptions}{%
\subsubsection{Trust Assumptions}\label{sec:04-Frontend__frontend-security-model:trust-assumptions}}

The frontend is treated as \textbf{fully untrusted}.

Assumptions:

\begin{itemize}
\tightlist
\item
  Client-side logic can be modified
\item
  Client-side validation can be bypassed
\item
  Requests may be forged or replayed
\end{itemize}

All security enforcement must occur server-side.

\begin{center}\rule{0.5\linewidth}{0.5pt}\end{center}

\hypertarget{token-handling}{%
\subsubsection{Token Handling}\label{sec:04-Frontend__frontend-security-model:token-handling}}

\hypertarget{access-tokens}{%
\paragraph{Access Tokens}\label{sec:04-Frontend__frontend-security-model:access-tokens}}

\begin{itemize}
\tightlist
\item
  Short-lived JWTs
\item
  Sent in Authorization headers
\item
  Never embedded in URLs
\end{itemize}

\hypertarget{refresh-tokens}{%
\paragraph{Refresh Tokens}\label{sec:04-Frontend__frontend-security-model:refresh-tokens}}

\begin{itemize}
\tightlist
\item
  Not directly accessed by frontend JavaScript when possible
\item
  Used only for token renewal
\item
  Validated server-side
\end{itemize}

Exact storage strategy may vary by frontend implementation.

\begin{center}\rule{0.5\linewidth}{0.5pt}\end{center}

\hypertarget{recommended-token-storage-strategies}{%
\subsubsection{Recommended Token Storage Strategies}\label{sec:04-Frontend__frontend-security-model:recommended-token-storage-strategies}}

\hypertarget{in-memory-storage}{%
\paragraph{In-Memory Storage}\label{sec:04-Frontend__frontend-security-model:in-memory-storage}}

\begin{itemize}
\tightlist
\item
  Reduced XSS exposure
\item
  Token lost on refresh
\end{itemize}

\hypertarget{secure-cookies-http-only}{%
\paragraph{Secure Cookies (HTTP-only)}\label{sec:04-Frontend__frontend-security-model:secure-cookies-http-only}}

\begin{itemize}
\tightlist
\item
  Strong XSS protection
\item
  Requires CSRF protections
\end{itemize}

LocalStorage is discouraged for sensitive tokens.

\begin{center}\rule{0.5\linewidth}{0.5pt}\end{center}

\hypertarget{client-side-validation}{%
\subsubsection{Client-Side Validation}\label{sec:04-Frontend__frontend-security-model:client-side-validation}}

Client-side validation is used for:

\begin{itemize}
\tightlist
\item
  UX improvement
\item
  Early feedback
\item
  Preventing obvious mistakes
\end{itemize}

Client-side validation is \textbf{never authoritative}.

\begin{center}\rule{0.5\linewidth}{0.5pt}\end{center}

\hypertarget{cors-and-browser-security}{%
\subsubsection{CORS and Browser Security}\label{sec:04-Frontend__frontend-security-model:cors-and-browser-security}}

The frontend relies on backend CORS configuration.

Security considerations:

\begin{itemize}
\tightlist
\item
  Strict origin configuration
\item
  Limited allowed methods
\item
  Controlled headers
\end{itemize}

Browser-enforced policies complement backend controls.

\begin{center}\rule{0.5\linewidth}{0.5pt}\end{center}

\hypertarget{error-and-failure-safety}{%
\subsubsection{Error and Failure Safety}\label{sec:04-Frontend__frontend-security-model:error-and-failure-safety}}

Frontend behavior under failure:

\begin{itemize}
\tightlist
\item
  Does not assume safety when errors occur
\item
  Displays partial results explicitly
\item
  Warns users when intelligence data is unavailable
\end{itemize}

\begin{center}\rule{0.5\linewidth}{0.5pt}\end{center}

\hypertarget{prevented-attack-classes}{%
\subsubsection{Prevented Attack Classes}\label{sec:04-Frontend__frontend-security-model:prevented-attack-classes}}

By design, the frontend does not allow:

\begin{itemize}
\tightlist
\item
  Privilege escalation via UI
\item
  Direct external intelligence API access
\item
  Trust decisions based on client state
\end{itemize}

\begin{center}\rule{0.5\linewidth}{0.5pt}\end{center}

\hypertarget{security-responsibilities-summary}{%
\subsubsection{Security Responsibilities Summary}\label{sec:04-Frontend__frontend-security-model:security-responsibilities-summary}}

\begin{longtable}[]{@{}ll@{}}
\toprule
Responsibility & Location \\
\midrule
\endhead
Authentication & Backend \\
Authorization & Backend \\
Risk Scoring & Backend \\
Input Validation & Backend \\
Secret Storage & Backend \\
Visualization & Frontend \\
\bottomrule
\end{longtable}

\begin{center}\rule{0.5\linewidth}{0.5pt}\end{center}

\hypertarget{security-model-summary}{%
\subsubsection{Security Model Summary}\label{sec:04-Frontend__frontend-security-model:security-model-summary}}

The frontend enhances usability while preserving strict trust boundaries.

Compromise of the frontend does not imply compromise of DefendX's security logic.
